\selectlanguage{english}
\begin{abstract}
    \noindent Az \emph{Alkalmazott fizikai módszerek laboratórium} első alkalmával a gamma-spektroszkópia témakörét jártuk körül a labormunka során, melyen különböző radioaktív elemeket tartalmazó minták gamma-spektrumait vizsgáltuk. A labor ideje alatt megismerkedtünk a méréshez használt berendezéssel, egy germánium félvezető detektorral, és a hozzá tartozó eszközökkel. Felügyeltük a detektor előzetes beállítását és elvégeztük a rendszer energiakalibrációját, majd egyéni mérőfeladatokat hajtottunk végre a laborvezető utasításai alapján. Végezetül meghatároztuk ezek esetében a detektor hatásfokát is. \\
    Az egyéni feladat során egy ismeretlen, sárga színű, kristályos felületű anyag gamma-spektrumának vizsgálatát kellett elvégeznem.
\end{abstract}
\selectlanguage{magyar}

\begin{multicols}{2}
\section{Bevezetés}
A gamma-spektroszkópia módszere arra a megfigyelésre alapul, miszerint a $\gamma$-bomlásra képes magok karakterisztikus hullámhosszú, és így meghatározott energiájú fotonokat bocsájtanak ki magukból, mikor gerjesztett állapotból egy alacsonyabb állapotba kerülnek. Ezek a fotonok egy detektorban képesek leadni az energiájukat, amely energiát képesek vagyunk detektálni. \newline
A labormunka során egy nagy tisztaságú germánium félvezető detektor segítségével mértük ki különböző anyagok gamma-spektrumait, melyben az elhaladó nagyenergiás $\gamma$-fotonok - Compton-szórás, fotoeffektus, vagy párkeltés során - ionizálják a környezetüket, ezzel elektron-lyuk párokat keltve a félvezetőben. A detektorra kapcsolt feszítőfeszültség ezeket a kialakult párokat eltávolítja egymástól, megakadályozva azok gyors rekombinációját, elődiézve egyúttal egy áramimpulzus létrejöttét a félvezetőben. Ez az impulzus detektálható és a foton által leadott energiával arányos. Sok hasonló foton energiájának megmérésével megkapjuk a minta pontos gamma-spektrumát, mely alapján a mintában található elemek beazonosítható válnak.


\section{Energiakalibráció}
A mérés során nem közvetlenül a detektorba csapódó fotonok energiáját, hanem az érzékeny, félvezető részében ionizáció hatására létrejövő áramimpulzusok nagyságát vagyunk képesek mérni. Ezeket az impulzusokat egy ún. \q{amplitúdó-analizátor} folyamatosan rögzíti. Az analizátor feladata, hogy egy adott áramerősség tartományt egyenlő szélességű \emph{bin}ekre felosszon, majd számolja, hogy a mérés ideje alatt minden tetszőleges $\left[ I,\ I + \delta I \right]$ \emph{bin}be hány darab áramimpulzus érkezik a detektorból. Ennek a jele egy számítógépre van kötve, melyen egy mérőszoftverrel valós időben követhetjük ezen hisztogram fejlődését. Az adott minta aktivitásától függően, viszonylag rövid idő alatt már jól felismerhető válik a gamma-spektrum ismert alakja, a mintában található elemekre jellemző karakterisztikus foto-csúcsokkal együtt. \newline
Az energiakalibráció során egy olyan ismert spektrumú anyagot helyezünk a detektorba, melynek karakterisztikus csúcsait könnyen azonosítani tudjuk és ismerjük a pontos energiájukat. Ezzel viszonylag pontosan meg tudjuk határozni az analizátorunk skálázását és annak \emph{bin} -- energia függvényét. Ezt követően már egy ismeretlen mintából származó csúcsokhoz tartozó energiákat is meg tudjuk mondani. \newline
Esetünkben ezt a kalibrációt egy $^{232}$Th tartalmú, Auer-gázégő izzóharisnyájával végeztük el. A mérést $t = 30\ \text{s}$ hosszan végeztük, mely során $N = 3473$ áramimpulzust észleltünk. Az általunk használt analizátor névlegesen $4096$ \emph{bin}-re bontja fel a mért áramimpulzus-spektrumot, azonban a mi mérésünk során csak $4083$ \emph{bin} működött, feltehetően a műszer hibájából fakadóan. Néhány karakterisztikus csúcs azonosítása után a mérőszoftver segítségével elvégeztük a koordináta-tengely transzformációt, mely \emph{bin} helyiértékekből keV mértékegységbe váltja át a skálánkat. Az alábbi összefüggést kaptuk az illesztés után:
\begin{equation}
    E
    =
    0.9255 * \text{BIN} - 1.7343
\end{equation}
ahol $\left[ E \right] = \text{keV}$, BIN pedig az adott \emph{bin} helyiértéke $0$ és $4083$ között. Ezáltal az X-tengelyen az energiaskála $E = -1.7343$ keV és $E = 3777.0822$ keV értékek között húzódott, magyarán kb. $2$ keV-el mértünk minden helyen többet, mint a valós érték. Megjegyzendő, hogy ez nem a minden helyen vett pontos eltérés, hanem csak egy közelítő becslés, mely igazán csak a kalibráció során az illesztéshez használt foto-csúcsaink közötti szakaszra megfelelő.


\section{A hatásfok mérése}
A detektorba érkező fotonok nem $100$\%-os valószínűséggel adják le az energiájukat, hanem vannak esetek, amikor a foton egyszerűen kölcsönhatás nélkül áthalad azon. Az összes mért és összes detektor felé kibocsájtódott foton arányát hívjuk a detektor hatásfokának, melyet $\eta$-val jelölünk. Ennek értéke az egyes energiatartományok más és más lehet, így minden foto-csúcsra egy individuális értéket fogunk kapni az $\eta$-ra eredményül. \newline
Ennek kiértékelését egy külön szoftverrel végeztük, mely a foto-csúcsok centroidjainak pontos energiájából, valamint a használt minta anyagi- és geoemtriai tulajdonságaiból számította ki számunkra az egyes foto-csúcsokhoz tartozó energiaértékeken mérhető hatásfokot. Ezeknek értékeit a \ref{table:3}. táblázatban közlöm.


\section{Egyéni feladat kiértékelése}
\subsection{A minta spektrumában található csúcsok azonosítása}
Az egyénileg történő mérés során egy kb. $1$ cm átmérőjű, sárga színű, kristályos külsejű kődarabot kellet megvizsgálnom. Az anyag gamma-spektrumát $t = 1073$ s hosszú mérés során vettem fel, mely során $N = 264077$ áramimpulzust észleltem. A teljes lemért spektrumot a \ref{fig:1}. ábrán illusztráltam, míg a \ref{fig:2}. és \ref{fig:3}. ábrán ugyanezen spektrum egyes, kinagyított részeit jelenítettem meg a részletgazdagság reményében. \newline
A mérőprogrammal $6$ foto-csúcsot vizsgáltam meg, melyeknek felvettem a csúcsterületüket, és azok hibáját, valamint meghatároztam a csúcs pontos, mért helyét. Olyan csúcsokat választottam ki, melyek valamelyik ismert, természetes bomlási családból származhatnak, és melyeket - egy kivételével - könnyen azonosítani is tudtam. Ezen csúcsok pontos képét a \ref{fig:4}. - \ref{fig:9}. ábrákon vizualizáltam. \newline
A $6$ lemért csúcs közül csak $5$ darab pontos adatait sikerült egyértelműen, $\gamma$-energiákat tartalmazó standardizált táblázatokból leolvasnom (\cite{firestone19978th} és \cite{lnbl_nuclear}). Ezek adatait energia szerinti növekvő sorrendben az \ref{table:1}. táblázatban gyűjtöttem össze. Az utolsó, az ábrákon \texttt{Unknown} felirattal ellátott, $764.24$ keV-es csúcs mibenlétére azonban ezen táblázatok alapján felállítható hipotézisek mindegyikét cáfolnom kellett. A \ref{table:2}. táblázatban azok a magok szerepelnek, melyek $50$ napnál nagyobb felezési idővel rendelkeznek és $760$ keV, valamint $766$ keV között bocsájtanak ki $\gamma$-fotonokat. Az $50$ napos alsó határ megvalásztása a laborban található minta korából fakadó döntés volt. A minta évek óta a laborban áll, így ennél alacsonyabb felezési idejű anyagok már biztosan teljesen elbomlottak benne. A lehetségesen szóbajövő magok közül azonban mindegyik esetén sok másik csúcsnak is meg kellett volna jelennie a spektrumban, ennél jelentősen nagyobb intenzitásokkal, melyek közül azonban egy sem volt megtalálható.

\subsection{Az aktivitás kiszámítása}
A mintában tallható radioaktív elemek $A$ aktivitását az alábbi képlet segítségével határozhatjuk meg:

\begin{equation}
A
=
\frac{S}{t * I \left( E, n \right) * \eta \left( E, n \right)}
\end{equation}
ahol $S$ az adott foto-csúcs alatti nettó terület, $t$ a mérés időtartama, $I \left( E, n \right)$ az adott energiájú $\gamma$-fotonok intenzitása, az úgynevezett \q{elágazási arány}, míg végül $\eta \left( E, n \right)$ a detektálási hatásfok \citep{gamma_ray_spec}. \newline
Ezen adatok mindegyike ismert és az (\ref{table:1}). és (\ref{table:3}). táblázatból kiolvasható. Az egyes csúcsokhoz tartozó aktivitásértékek hibái a standard hibaterjedés szabályai alapján az alábbi képlet segítségével számíthatóak ki:

\begin{equation}
\Delta A
=
\left(
\frac{\Delta S}{S} + \frac{\Delta t}{t} + \frac{\Delta I}{I} + \frac{\Delta \eta}{\eta}
\right)
\end{equation}
A számításokban az $I$ intenzitás és a $t$ időmérés hibáját $0$-nak vettem. A kapott értékeket a \ref{table:4}. táblázatban közöltem. Ezen adatok felhasználásával már megkapható a mintában található $^{235}$U mennyisége, mely a táblázatokban ugyanígy van jelölve, valamint az $^{238}$U mennyisége is, mely a táblázatban is jelölt leányelemvel, a $^{234}$Pa$^{m}$ maggal szekuláris egyensúlyban tartózkodik, tehát aktivitásuk megegyezik. Az aktivitás megadható a következő összefüggésből is:

\begin{equation}
A = \lambda N
\end{equation}
ahol $\lambda = \ln \left( 2 \right) /\ T_{1/2}$ a bomlási állandó, $N$ pedig a részecskeszám. Az utóbbi az $A$ és $\lambda$ ismeretében már megadható, mely értékeket szintén a \ref{table:4}. táblázatban közöltem. A részecskeszámok vizsgálatával kiszámíthatjuk az $^{235}$U és $^{238}$U arányát a mintában. Természetes körülmányek között az $^{235}$U részecskék aránya $0.72$\%, míg a maradék tulajdonképpen csak $^{238}$U. A jelenlegi arány a következő:

\begin{align}
C_{^{235}U}
&=
\frac{32.21 * 10^{16}}{32.21 * 10^{16} + 7.46 * 10^{18}}
=
\frac{3.22 * 10^{17}}{7.78 * 10^{18}}
\approx \nonumber \\
&\approx
0.0413
\end{align}

\begin{equation}
C_{^{238}U}
=
\frac{7.46 * 10^{18}}{7.78 * 10^{18}}
\approx
0.958
\end{equation}
Melynek relatív hibája megegyezik az $N$ hibájával, így határozottan kijelenthető, hogy a mintában található $^{235}$U / $^{238}$U arány kisebb, mint ahogy az a természetben előfordul, így feltehetően szegényített urántartalmú mintát vizsgáltam.

\section{Általános számítási feladatok}


\end{multicols}