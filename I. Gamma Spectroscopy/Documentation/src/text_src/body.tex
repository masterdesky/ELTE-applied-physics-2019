\section{Kérdések kidolgozása}
\begin{enumerate}
    \item Q: Milyen természetes radioaktív sorokat ismerünk?
    \begin{displayquote}
        A: $\alpha$-bomlás során az adott mag tömegszáma $4$-el csökken, $\beta$- és $\gamma$-bomlás során pedig nem változik. Emiatt 4 bomlási sort tudunk megkülönböztetni, melyeket egyértelműen a bennük található atomok tömegszámainak négyes maradéka különböztet meg egymástól. A Földön természetes körülmények között jelenleg is előforduló családok az $^{238}$U, $^{235}$U és $^{232}$Th. Rövid felezési ideje miatt már csak mesterséges folyamatokból ismert a $^{237}$Np család. A $^{232}$Th család atomjainak tömegszámai $4$-el osztva $0$ maradékot adnak eredményül, a $^{235}$U család esetében $1$-et, a $^{238}$U család tagjai $2$-t, míg végül a $^{237}$Np család tagjai 3-at.
    \end{displayquote}
    
    \item Q: Hogyan, mi alapján határozzuk meg a gránit urántartalmát?
    \begin{displayquote}
        A: A labor során ezt az ún. \q{gamma-spektroszkópia} módszerével határozzuk meg. Látni fogjuk, hogy a gránitot tulajdonképpen csak az $^{238}$U és annak leányelemeiből származó $\gamma$-bomlás során keletkező fotonok hagyják el, a benne található $^{232}$Th-ből származó fotonok csak alig érzékelhetők. Ezek energiáját megmérve, karakterisztikus csúcsokat azonosíthatunk a kibocsájtott spektrumban, melyből következtethetünk a minta radioaktív összetételére. A megfelelő energiacsúcsok alatti terület arányos az aktivitással (és a mérési idővel), így a keresett radioaktív elem koncentrációjával.
    \end{displayquote}
    
    \item Q: Hogyan működik, és milyen típusú az általunk használt detektor?
    \begin{displayquote}
        A: A mérésleírás alapján \q{a detektor egy nagy tisztaságú germánium félvezető detektor}. Ebben a mintából kisugárzódó $\gamma$-fotonok teljes energiája leadódhat valamilyen fény-anyag kölcsönhatás (fotoeffektus, Compton-szórás, vagy párkeltést követő annihiláció) során. Ez minden esetben valamilyen töltött részecske (vagy részecskék) kiválását, vagy keletkezését okozza a fotont elnyelő anyagban. A nagy tisztaságú félvezetőből készült detektorunk esetében ez elektron-lyuk párok keletkezését fogja jelenteni, mely elektronok nagy intenzitású áramlását eredményezi így a detektorban, mely áramimpulzust arányos az elnyelődött foton energiájával és mely impulzust már meg tudunk mérni.
    \end{displayquote}
    
    \item Q: Lehetne-e a fenti detektorral alfa- illetve béta-sugárzást mérni, és miért?
    \begin{displayquote}
        A: A detektor minden esetben a benne található félvezetőben keletkező elektron-lyuk párok keletkezését érzékeli. Minden olyan folyamat, melyek során ilyenek keletkeznek, azt a detektor képes érzékelni. Az $\alpha$-bomlás során egy nagy tömegű, töltött $\alpha$-részecske hagyja el a bomló magot, míg $\beta$-bomlás során egy $e^{-}$, vagy egy $e^{+}$ távozik. Ezek a töltésüknek köszönhetően, kölcsönhatva a félvezető töltött részecskéivel, fel tudják borítani a félvezetőben található egyensúlyi helyzetet, ezzel áramimpulzust létrehozva. Természetesen ezen részecskék szabad úthossza jóval rövidebb, mint a $\gamma$-bomlás során létrejövő fotoné, így detektálásuk minden esetre jóval nehézkesebb.
    \end{displayquote}
    
    \item Q: Ha $1$ kg talajban $0,01$\% uránt találunk, hogyan kell kiszámítani az urán aktivitását?
    \begin{displayquote}
        A: Az aktivitás képlete $A = \lambda N$, ahol $N$ a részecskeszám, $\lambda$ pedig a bomlási állandó. Ez esetben
        \begin{equation}
            A
            =
            \lambda_{^{238}U}
            *
            \frac{m_{^{238}U}}{M_{^{238}U}}
            *
            N_{A}
            =
            6 * 10^{-18}\ \frac{1}{\text{s}}
            *
            \frac{1000 * 0.001\ \text{g}}{238,03\ \frac{\text{g}}{\text{mol}}}
            *
            6,022 * 10^{23} \frac{1}{\text{mol}}
            \approx
            15180\ \text{Bq}
        \end{equation}
    \end{displayquote}
    
    \item Q: Milyen adatok kellenek a mérésünkben az aktivitás kiszámításához? (képlet is)
    \begin{displayquote}
        A: A mérés során használt módszerben a $\gamma$-foton beütések energiaspektrumán található egyik karakterisztikus csúcs alatti nettó területet ($N$), a detektor ezen energiaintervallumra vonatkozó hatásfokát ($\eta$), valamint az ilyen energiájú $\gamma$-fotonok intenzitását ($I$) tudjuk megmérni. Ezek mellett ismert érték a mérési idő ($t$), mellyel az aktivitás fordítottan arányos. Ezek alapján az aktivitást a következő képlet alapján határozhatjuk meg:
        \begin{equation}
            A
            =
            \frac{N}{\eta * I * t}
        \end{equation}
    \end{displayquote}
    
    \item Q: Miért van szükség nagyfeszültségre a germánium detektor használatakor?
    \begin{displayquote}
        A: A mérési összeállításban használt $3000$ - $4000$ V-os feszítő-feszültség szerepe, hogy megakadályozza a $\gamma$-fotonok energialeadása során keletkező elektron-lyuk párok rekombinációját, rákényszerítve a félvezető elektronjait, hogy a pozitív-, a félvezető elektronhiányait pedig, hogy a negatív elektródákra gyűljenek össze és ezzel egy áramimpulzust hozzanak létre.
    \end{displayquote}
    
    \item Q: Miért kell a detektort hűteni?
    \begin{displayquote}
        A: Ha nem hűtenénk folyamatosan a detektort, akkor nagy hőmérsékleten a félvezetőre kapcsolt feszítő-feszültség miatt már radioaktivitás nélkül is folyamatos áram folyna benne, így folyamatosan erős termikus zaj zavarná a mérésünket. Ezt elkerülendő, a detektor egy rézrúdra van helyezve, melynek egyik vége folyamatosan $-196^{\circ}$C hőmérsékletű folyékony nitrogénbe van mártva.
    \end{displayquote}
    
    \item Q: $0,119$ g tiszta $^{238}$U-nak mekkora az aktivitása, ha a bomlási állandója (kerekítve) $6 * 10^{-18}$ s$^{-1}$?
    \begin{displayquote}
        A: Az aktivitás képlete $A = \lambda N$, ahol $N$ a részecskeszám, $\lambda$ pedig a bomlási állandó. Ez esetben
        \begin{equation}
            A
            =
            \lambda_{^{238}U}
            *
            \frac{m_{^{238}U}}{M_{^{238}U}}
            *
            N_{A}
            =
            6 * 10^{-18}\ \frac{1}{\text{s}}
            *
            \frac{0,119\ \text{g}}{238,03\ \frac{\text{g}}{\text{mol}}}
            *
            6,022 * 10^{23} \frac{1}{\text{mol}}
            \approx
            1806\ \text{Bq}
        \end{equation}
    \end{displayquote}
    
    \item Q: Mi a szekuláris egyensúly, és mi a feltétele?
    \begin{displayquote}
        A: Ha egy bomlási sorban az anyaelem felezési ideje nagyságrendekkel hosszabb, mint a keletkező leányelemekéé, akkor a kettőjük aktivitása megegyezik, hisz értelemszerűen rövid időskálán pontosan annyi leánymag bomlik el, mint amennyi az anyaelem bomlása során keletkezett. Pontosan ezt az állapotot nevezzük \emph{szekuláris egyensúly}nak, amikor egy bomlási sorban a keletkező leányelemek aktivitása megegyezik az anyaelem aktivitásával. Ilyenkor $A = \lambda_{1} N_{1} = \lambda_{2} N_{2} = \dots = \lambda_{i} N_{i}$.
    \end{displayquote}
    
    \item Q: Hogyan működik az amplitúdó-analizátor, mi a feladata?
    \begin{displayquote}
        A: Az analizátor feladata, hogy a energiaspektrumot $8192$ db egyenlő széleségű \emph{bin}re felossza és egy mérés során számlálja, hogy hány darab $\varepsilon$ energiájú $\gamma$-foton nyelődik el a detektorban az adott \emph{bin} által lefedett $\varepsilon \in \left[ E,\ E + \delta E \right]$ energiatartományban. Végeredményül az elnyelődött fotonok energiájának eloszlását (hisztogramját) kapjuk meg.
    \end{displayquote}
    
    \item Q: Hogyan kalibráljuk a mérési elrendezést (energiakalibráció)?
    \begin{displayquote}
        A: Egy $^{232}$Th izotóp felhasználásával, melynek $\gamma$-emisszióját pár percen keresztül mérjük. Ezzel az ismert energiájú $\gamma$-fotonokat kibocsájtó anyag segítségével meghatározhatjuk, hogy az egyes karakterisztikus csúcsok pontosan mely \emph{bin}ekbe esnek az amplitúdó-analizátoron, így feltérképezve a mérőműszerünk skálázását. Ezután már könnyedén kiszámíthatjuk, hogy egy ismeretlen izotóp által létrehozott csúcsokhoz pontosan milyen energiák tartoznak, így azonosítva az ismeretlen anyagot.
    \end{displayquote}
    
    \item Q: Lehetne-e a fenti detektorral béta-sugárzást mérni, és miért?
    \begin{displayquote}
        A: Erre ugyanaz a válasz, mint a $4$)-es pontban.
    \end{displayquote}
    
    \item Q: Hogyan hat kölcsön a detektorral a beérkező gamma-sugárzás, és hogyan függ ez az energiától?
    \begin{displayquote}
        A: Erre ugyanaz a válasz, mint a $3$)-as pontban.
    \end{displayquote}
    
    \item Q: Két csúcsot találunk a spektrumban, amelyek ugyanahhoz az izotóphoz tartoznak. Mindkettőre kiszámoljuk az aktivitást. Az egyikre $100 \pm 10$ Bq, a másikra $112 \pm 5$ Bq az eredmény. Mennyi a két eredmény súlyozott átlaga, és annak hibája?
    \begin{displayquote}
        A: Az aktivitás súlyozott átlaga megadható az alábbi képlet segítségével:
        \begin{equation}
            \left< A \right>
            =
            \frac{\sum_{i} \dfrac{A_{i}}{\sigma_{i}^{2}}}{\sum_{i} \dfrac{1}{\sigma_{i}^{2}}}
        \end{equation}
        Ahol $\sigma_{i}$ az adott $A_{i}$ aktivitás abszolút hibája. Ebből a fenti értékeket behelyettesítve kapjuk meg a választ a kérdésben szereplő problémára:
        \begin{equation}
            \left< A \right>
            =
            \left( \frac{100}{10^{2}} + \frac{112}{5^{2}} \right)
            *
            \left( \frac{1}{10^{2}} + \frac{1}{5^{2}} \right)^{-1}\ \text{Bq}
            =
            109,6\ \text{Bq}
        \end{equation}
        Míg ennek \emph{abszolút} hibája megkapható szabály szerint a következő képletből:
        \begin{equation}
            \sigma_{\left< A \right>}
            =
            \frac{1}{\sqrt{\sum_{i} \dfrac{1}{\sigma_{i}^{2}}}}
        \end{equation}
        Melyet felhasználva megkapjuk a szükséges hibaértéket is:
        \begin{equation}
            \sigma_{\left< A \right>}
            =
            \frac{1}{\sqrt{\dfrac{1}{10^{2}} + \dfrac{1}{5^{2}}}}
            =
            4,5\ \text{Bq}
        \end{equation}
        Így az aktivitás végleges értéke:
        \begin{equation*}
            \left< A \right>
            =
            \left(109,6 \pm 4,5 \right)\ \text{Bq}
        \end{equation*}
    \end{displayquote}
    
    \item Q: A nettó csúcsterületre az eredményünk $200 \pm 10$, a hatásfok pedig $0,02 \pm 10$\%. Az intenzitás az adott vonalra $0,5$ és $20$ másodpercig mértünk. Mekkora az izotóp aktivitása a mintában?
    \begin{displayquote}
        A: Már a $6$)-os pontban ismertetett képlet alapján kiszámíthatjuk hibával együtt az aktivitást:
        \begin{align}
            A
            &=
            \frac{N}{\eta * I * t}
            =
            \frac{200 \pm 5\%}{\left( 0,02 \pm 10\% \right) * 0,5 * 20\ \text{s}}
            =
            \left( \frac{200}{0,02 * 0,5 * 20} \pm \left( 100 * \sqrt{0,05^{2} + 0,1^{2}} \right)\% \right)\ \frac{1}{\text{s}}
            = \nonumber \\
            &=
            \left( 1000 \pm 11,8\% \right)\ \text{Bq}
        \end{align}
    \end{displayquote}
    
    \item Q: Mi az önárnyékolás és milyen nehézséget okoz a mérésnél?
    \begin{displayquote}
        A: Egy kiterjedt minta esetén annak a detektorhoz közeli részeiből jóval nagyobb intenzitással detektálhatunk $\gamma$-fotonokat, mint távolabbról. Ennek oka, hogy a távoli részek egyrészt jóval kisebb térszögben látják a detektort, másrészt hosszabb anyagtömegen kell áthatoljanak, hogy egyáltalán elérjék a detektort, abszorpció nélkül.
    \end{displayquote}
    
    \item Q: Miért kell tudni a minta kémiai összetételét ahhoz, hogy meghatározhassuk az urántartalmát?
    \begin{displayquote}
        A: Az urántartalom pontos meghatározásához a számításainkat korrigálnunk kell a $17$)-es pontban említett önabszorpció mértékével. Ahhoz, hogy ennek nagyságát megbecsülhessük, ismernünk kell az anyagban található elemekre vonatkozó kölcsönhatások pontos rendszám- és energiafüggését.
    \end{displayquote}
    
    \item Q: A mintában a $^{214}$Bi aktivitására $1000 \pm 55$ Bq, míg a $^{226}$Ra aktivitása $1500 \pm 75$ Bq. Hogyan magyarázhatjuk a különbséget?
    \begin{displayquote}
        A: A bizmuttal ellentétben a radon képes a közeteken átdiffundálni, így az utóbbi esetében nagyobb sugárzást mérhetünk ebből a faktorból fakadóan.
    \end{displayquote}
    
    \item Q: Az egyik mintánkban csak az $^{235}$U gamma-sugárzása észlelhető, a $^{214}$Pb sugárzása nem. Mi lehet ennek az oka?
    \begin{displayquote}
        A: Az $1$)-es pontban ismertetetteknek megfelelően, a $^{214}$Pb nem tartozhat a $^{235}$U családjába, ugyanis tömegszámaik $4$-es maradéka nem egyezik meg. Ha eredetileg csak a mért $^{235}$U volt megtalálható az anyagban, akkor semmilyen módon nem kerülhetet oda $^{214}$Pb, ugyanis a $^{235}$U nem arra bomlik le. Mellesleg ismert, hogy a $^{214}$Pb a $^{238}$U család tagja.
    \end{displayquote}
    
    \item Q: Milyen mesterséges és természetes izotópok mutathatók ki könnyen egy talajmintából gamma-spektroszkópiával?
    \begin{displayquote}
        A: A természetes forrásokból származóan a bomlási sorokból származóan az $^{238}$U, $^{235}$U és $^{232}$Th detektálható, valamint az erősen $\beta$-bomló $^{40}$K is könnyen azonosítható. A légköri atomrobbantásokból és a csernobili atomkatasztrófából származóan a $^{137}$Cs izotóp is fellelhető a talajmintákban.
    \end{displayquote}
\end{enumerate}