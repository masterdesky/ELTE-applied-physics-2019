\selectlanguage{english}
\begin{abstract}
    \noindent Az \emph{Alkalmazott fizikai módszerek laboratórium} második alkalmával az optikai pumpálás módszerét jártuk körül, mely során egy $^{85}$Ru és $^{87}$Ru izotópokat tartalmazó rubídiumgázt sugároztunk be lézerrel. A labormunka során megmértük a pumpálás, valamint a 
\end{abstract}
\selectlanguage{magyar}

\begin{multicols}{2}
\section{Bevezetés}
Az optikai pumpálás alatt azt a folyamatot értjük, mely során lézer besugárzásával egy mintában található elektronokat egy magasabb energiaszintre gerjesztünk, ezzel populáció inverziót létrehozva, vagyis megfordítva az egyes energiaszintek, Boltzmann-eloszlásból várható betöltöttségi mértékét.


\end{multicols}