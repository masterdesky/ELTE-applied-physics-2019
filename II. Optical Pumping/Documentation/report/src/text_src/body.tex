\selectlanguage{english}
\begin{abstract}
    \noindent Az \emph{Alkalmazott fizikai módszerek laboratórium} második alkalmával az optikai pumpálás módszerét jártuk körül, mely során egy $^{85}$Ru és $^{87}$Ru izotópokat tartalmazó rubídiumgázt sugároztunk be lézerrel. A labormunka során egy Rb- és Kr-tartalmú kisülési cső segítségével megmértük a rendszerre jellemző $\tau = \left( 1/T_{p} + 1/T_{1} \right)^{-1}$, valamint a $T_{2}$ relaxációs időket. Ezt követően egy rádiófrekvenciás jelgenerátorral $4$ különböző frekvencián feltérképeztük a két rubídiumizotóphoz tartozó rezonanciaátmenetek pozícióját, mely során megmértük a Föld mágneses terének nagyságát is. Végül megpróbáltuk meghatározni a két rubídiumizotóphoz tartozó $g_{F}$-et -- a hiperfinom kölcsönhatást is figyelembevevő Landé-féle g-faktort -- mely azonban a laborban található eszköz műszaki hibájából fakadóan csupán az $I=3/2$ magspinnel rendelkező $^{87}$Rb izotópra sikerült.
\end{abstract}
\selectlanguage{magyar}

\begin{multicols}{2}
\section{Bevezetés}
Az optikai pumpálás alatt azt a folyamatot értjük, mely során fény (jelen esetben lézer) besugárzásával valamilyen mintában található elektronokat magasabb energiaszintre gerjesztünk és ún. \q{populáció inverziót} hozunk létre, vagyis megfordítjuk az egyes energiaszintek -- egyensúlyi helyzetben a Boltzmann-eloszlás alapján várható -- betöltöttségi arányait.

\section{A folyamatban szereplő időállandók}


\end{multicols}