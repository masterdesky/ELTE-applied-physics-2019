\selectlanguage{english}
\begin{abstract}
    \noindent Az \emph{Alkalmazott fizikai módszerek laboratórium} második alkalmával az optikai pumpálás módszerét jártuk körül, mely során egy $^{85}$Rb és $^{87}$Rb izotópokat tartalmazó rubídiumgázt sugároztunk be lézerrel, majd nagyfrekvenciás elektromágneses sugárzással. A labormunka során egy Rb- és Kr-tartalmú kisülési cső segítségével megmértük a rendszerre jellemző $\tau = \left( 1/T_{p} + 1/T_{1} \right)^{-1}$, valamint a $T_{2}$ relaxációs időket. Ezt követően egy rádiófrekvenciás jelgenerátorral $4$ különböző frekvencián feltérképeztük a két rubídiumizotóphoz tartozó rezonanciaátmenetek pozícióját, mely során megmértük a Föld mágneses terének nagyságát is. Végül megpróbáltuk meghatározni a két rubídiumizotóphoz tartozó a hiperfinom kölcsönhatást is figyelembe vevő Landé-féle g-faktort ($g_{F}$) és azok magspinjét, melyek azonban a laborban található eszköz műszaki hibájából fakadóan csupán az egyik izotópra sikerültek.
\end{abstract}
\selectlanguage{magyar}

\begin{multicols}{2}
\section{Bevezetés}
Az optikai pumpálás alatt azt a folyamatot értjük, mely során - optikai besugárzással - egy mintában található elektronokat magasabb energiaszintre gerjesztünk és és ezzel megváltoztatjuk az egyes energiaszintek (első sorban a Zeeman- valamint hiperfinom kölcsönhatás miatt kialakult alnívók) -- egyensúlyi helyzetben és kellően magas hőmérsékleten a Boltzmann-eloszlás alapján várható -- betöltési arányait. A pumpálás során célunk, hogy ún. \q{populáció inverziót} hozzunk létre az általunk gerjesztett két nívó között. Ezalatt az az állapotot értjük, amikor a rendszerben található részecskék közül több tartózkodik a magasabb energiaszinten, mint amennyi az alacsonyon. A gerjesztéssel feltöltött energiaszint emiatt nem tartózkodik egyensúlyi állapotban és erősen metastabil, így arról folyamatosan elektronok fognak spontán emisszió útján az alacsonyabb szintekre vándorolni, miközben az energiakülönbséggel megegyező energiájú, koherens fotonokat bocsájtanak ki. A populáció inverzió egyik alapfeltétele emiatt, hogy a rendszerünk legalább $3$ db nívóval rendelkezzen, melyeket jelöljünk $A$, $B$ és $C$ indexekkel. Ha ekkor $E_{B} > E_{C} > E_{A}$, akkor pumpálás során az $A \to B$ átmenetet próbáljuk megvalósítani, melyről spontán emisszió útján a $C$ nívóra fognak az egyes részecskék leugrani. Ezzel effektíve az $A$ és $C$ nívók közti populáció inverziót tudjuk megvalósítani egy köztes ($B$-re történő) gerjesztés felhasználásával \citep{kubo1972optical}. \par
A mérésben rubídiumgáz részecskéinek két, Zeeman-felhasadásból származó nívója között valósítottunk meg populáció inverziót, ahol a harmadik (köztes) nívót az egyik hiperfinom-kölcsönhatásból származó energiaszinttel biztosítottuk. A rubídiumatomok Zeeman-felhasadását egy Helmholtz-terekcs felhasználásával valósítottuk meg.


\section{Mérési körülmények}
A labor feladatainak elvégzése során két módszer segítségével gerjesztettük a rubídiumgáz. Első esetben egy Rb- és Kr puffer gázt tartalmazó kisülési cső fényével, melyet egy optikai rendszeren keresztül vezettünk a mintára. A kisülési csőböl kilépő fény sugármenetét egy gyűjtőlencse alakította ki, melyet ezután egy, a Rb hevítéséből származó D$_{1}$ vonalat átengedő, $\lambda_{\text{max}} = 795$ nm-es interferenciaszűrővel tettünk közel koherenssé. A sugár ezután áthaladt egy polarizációs lencsén, mely lineárisan poláros fényt ezt követően egy kettőstörő kristállyal (ún. $\lambda/4$-es lemezzel) cirkulárisan polarizálttá változtattunk. A mintán történő áthaladás után a sugarat egy újabb lencserendszerrel összegyűjtjük, majd egy fotodiódára vezetjük rá, melynek jelét egy erősítőn keresztül egy oszcilloszkópon észlelhetjük. \par
Második esetben egy rádiófrekvenciás jelgenerátor jelét kapcsoltuk egy második, a Zeeman-felhasadást biztosító tekercspárra merőleges Helmholtz-tekercsre, mellyel egy újabb, periodikus mágneses teret hoztunk létre, miközben továbbra is folytattuk a rendszer optikai pumpálását. Ha ez a második tér olyan $B$ indukcióval és $\nu$ frekvenciával rendelkezik, melyre teljesül a laborleírásban is ismertetett rezonanciafeltétel

\begin{equation} \label{eq:1}
E_{2} - E_{1}
=
h \nu
=
\mu_{B} * g_{F} * B
\end{equation}
akkor ez a besugárzás képes újabb átmeneteteket létrehozni a két Zeeman-nívó között. Ez lehet a felső szintről az alsóra történő indukált emisszió, vagy a másik irányba végbemenő abszorpció. A két nívó azonos betöltöttsége esetén mindkét átmenet egyenlő valószínűséggel fordul elő, azonban pumpálás során már nincs meg ez az egyensúly. Mivel a fenti nívó betölttötsége jóval nagyobb, így rádiófrekvenciás gerjesztés hatására a felsőbb energiaszintről az alsóbbra történő ugrás valószínűsége megnő. Végeredményben ez az extra gerjesztés az optikai pumpálás ellen fog dolgozni.

\section{A folyamatot meghatározó időállandók mérése}
Feladataink közé tartozott azt optikai pumpálás során lejátszódó folyamatokra jellemző $\tau$, valamint $T_{2}$ karakterisztikus időállandók mérése. A $T_{p}$ és $T_{1}$ állandókkal, melyek sorrendben a pumpálás bekapcsolásának és kikapcsolásának relaxációját írják le, nem foglalkoztunk, ugyanis ezek meghatározása jóval hosszabb mérési időt vett volna igénybe.
\subsection{A $\tau$ időállandó}
A $\tau$ állandó a mágneses tér bekapcsolása -- és így a Zeeman-effektus létrehozása -- utáni átmenet idejét hivatott leírni. A Zeeman-felhasadás létrejöttekor a populáció inverzió feltétele teljesül, így a pumpálás hatására elkezdődik az energiaszintek betöltöttségének átrendeződése, mely egy időben véges hosszúságú folyamat. A mágneses tér bekapcsolásakor a fotodióda jelén egy exponenciális felfutó élt láthatunk, mely az alábbi paraméteres függvénnyel írható le:

\begin{equation} \label{eq:2}
y
=
A + B * e^{C * t}
\end{equation}
ahol $\tau = 1/C$. Ezen élek könnyebb meghatározásához az első számú Helmholtz-tekercsekbe egy négyszögjelet vezettünk, mely így egy periodikus, fűrészfogra hasonlító jelet eredményezett a fotodiódán (ld. \ref{fig:1}. ábra). Kezdeti zavarunkban erről véletlenül két adatsort is elmentettünk, ezért úgy döntöttem, hogy mindkettőt feldolgozom, majd a kapott értékeket összehasonlítom és átlagolom, a pontosabb eredmény reményében. \newline
A két különálló adatsorból egy-egy kivágott felfutó élt vizsgáltam a továbbiakban. A két kivágott él a \ref{fig:2}. ábrán látható. Hogy az esetleges illesztési hibákat kiszűrjem, úgy döntöttem, hogy a jelet átengedem egy aluláteresztő szűrőn, melynek $f_{\text{C}}$ levágási frekvenciáját az eredeti jelalak Fourier-transzformációval kapott frekvenciaterét vizsgálva állapítottam meg (ld. \ref{fig:3}. ábra). Az $f_{\text{C}}$ értékét végül $f_{\text{C}} = 1000$ Hz-nek választottam, mely szűrőn átvezetett teljes jelsorozatok és kivágott éleik sorrendben a \ref{fig:4}. és \ref{fig:5}. ábrán láthatóak. A szűrés utáni frekvenciateret a \ref{fig:6}. ábrán vizualizáltam, melyen bejelöltem a szűrő levágási frekvenciájának értékét is a jobb átláthatóság kedvéért. \par
A kivágott jelalakokra történő illesztést előkészítendő, először megbecsültem az $A$, $B$ és $C$ paraméterek közelítő értékeit. Ehhez a közelítéshez a jeleket eltoltam a $t = 0$ pozícióba, majd az alábbi becsléseket adtam az egyes paraméterek számára (feltételezve közben, hogy $t = \infty$-ben az illesztett függvény jól láthatóan egy konstanshoz tart, tehát $C < 0$ minden esetben):

\begin{equation*}
A
=
\left. A + B * e^{C * t} \right|_{t = \infty} \quad (C < 0)
\end{equation*}
\begin{equation*}
B
=
\left[ \left. A + B * e^{C * t} \right|_{t = 0} \right] - A
\end{equation*}
\begin{equation*}
C
=
\left. \ln \left( \frac{f \left( t \right) - A}{B} \right) * \frac{1}{t} \right|_{t > 0; t < \infty}
\end{equation*}
Ahol:

\begin{equation}
f \left( t \right)
=
\left. A + B * e^{C * t} \right|_{t > 0;\ t < \infty}
\end{equation}
Végezetül a \ref{eq:2}. egyenletben is szereplő függvényt illesztettem a \texttt{Python}, \texttt{scipy} könyvtárának \texttt{curve\_fit} függvényével, mely illesztési iterációk kezdeti paraméterének az előbb felsorolt $A$, $B$ és $C$ értékeket adtam meg. \par
Az illesztések kezdeti paraméterei mind az eredeti, mind az aluláteresztő szűrőn átengedett jelek esetében a \ref{fig:7}. és \ref{fig:9}. ábrákon láthatóak. A véglegesen, optimalizált értékeket ugyanígy a \ref{fig:8}. és \ref{fig:10}. ábrákon ábrázoltam. \par
Végül a a $\tau$ értékére a \ref{table:1}. táblázatban láthatóakat kaptam a két mérés alapján.

\begin{center}
\begin{tabular}{|c|c|}
\hline
Mérés              & $\tau$ [s]                                \\ \hline \hline
Eredeti 1          & $2,60107 * 10^{-3} \pm 9,32279 * 10^{-5}$ \\ \hline
Eredeti 2          & $2,90588 * 10^{-3} \pm 7,08765 * 10^{-5}$ \\ \hline
Szűrt 1            & $2,47563 * 10^{-3} \pm 1,13762 * 10^{-5}$ \\ \hline
Szűrt 2            & $2,78519 * 10^{-3} \pm 8,73934 * 10^{-6}$ \\ \hline \hline
Eredeti (átl.)     & $2,75348 * 10^{-3} \pm 1,19374 * 10^{-4}$ \\ \hline
Szűrt (átl.)       & $2,63041 * 10^{-3} \pm 1,46366 * 10^{-5}$ \\ \hline
\end{tabular}
\captionof{table}{Az illesztések után a $\tau$ értékére kapott eredmények.} \label{table:1}
\end{center}

\subsection{A $T_{2}$ időállandó}
A $T_{2}$ időállandó a mágneses tér kikapcsolása utáni lecsengést hivatott leírni. A tér kikapcsolásával a Zeeman-nívók megszűnnek, így a pumpálás szükséges feltétele megszűnik. Ekkor a rendszer a $T_{1}$ időállandóval jelölt folyamatnál gyorsabban relaxál a termikus egyensúly felé, ugyanis ilyenkor a pumpálás csak felerősíti a kikapcsoláskor jobban telített felső energiaszintről az alsóbbra történő emissziót. Ez a fotodióda jelében egy lecsengő élet fog eredményezni. Gyakorlatban ebben az esetben nem szüntetjük meg teljesen a tekercs mágneses terét, hanem pontosan akkorát hozunk létre benne, amely ellensúlyozza a Föld mágneses teréből származó komponenst, így hozva létre effektíve zérus mágneses teret a minta helyén. \par
A mágneses tér és a fotodióda jele a \ref{fig:11}. ábrán látható. A jelalak feldolgozása és a $T_{2}$ meghatározása a $\tau$ esetén is leírtakkal megegyező módon történt. Az eredeti jelet itt is egy $f_{\text{C}} = 1000$ Hz levágási frekvenciájú aluláteresztő szűrőn engedtem át és az eredeti jellel azonos módon vizsgáltam. A szűrt jelalak a \ref{fig:12}. ábrán látható, míg a kiemelt eredeti és szűrt lefutó él sorrendben a \ref{fig:13}. és \ref{fig:14}. ábrákon találhatóak. \par
Az illesztés jelen esetben nem volt triviális feladat, ugyanis a lefutó él exponenciális részének alsó határa nehezen volt behatárolható, melyet a \texttt{curve\_fit} függvény így nem kezelt jól. A probléma megoldásaként megvizsgáltam, hogy az illesztés hibája hogyan változik a lefutó él végpontjainak különböző megválasztására (ld. \ref{fig:15}. ábra). Végül a legkisebb hibával rendelkező végpontot választottam az illesztéshez (ezek a kivágott jelalakok láthatóak a \ref{fig:13}. és \ref{fig:14}. ábrákon). Az illesztés során végül a \ref{fig:16} -- \ref{fig:19} ábrákat kaptam eredményül, ahol a \ref{fig:16}. és \ref{fig:18}. ábrákon a kezdeti paraméterekkel történő illesztett exponenciális, míg a \ref{fig:17}. és \ref{fig:19}. ábrákon az optimalizált paraméterekkel történő függvény van ábrázolva. \par
A $T_{2}$ értékére végül a \ref{table:2}. táblázatban található eredményeket kaptam.

\begin{center}
\begin{tabular}{|c|c|}
\hline
Mérés            & $\tau$ [s]                                \\ \hline \hline
Eredeti          & $5,16741 * 10^{-4} \pm 2,33856 * 10^{-5}$ \\ \hline
Szűrt            & $5,25913 * 10^{-4} \pm 5,01825 * 10^{-6}$ \\ \hline
\end{tabular}
\captionof{table}{Az illesztések után a $T_{2}$ értékére kapott eredmények.} \label{table:2}
\end{center}

\section{Rádiófrekvenciás gerjesztés}
A II. részben tárgyaltaknak megfelelően, a fenti mérések elvégzése után egy második, az elsőre merőleges tengelyű Helmholtz-tekercset kapcsoltunk be, melynek segítségével rádiófrekvenciás mágneses térrel sugároztuk be a rubídiumgáz mintánkat. A hozzákapcsolt jelgenerátor több különböző frekvenciájú jelet volt hivatott kiadni magából, azonban a műszer hibájából fakadóan csak $2$ különböző frekvenciát volt képes kibocsájtani. \par
A méréseket négy különböző csatornán próbáltuk elvégezni, melyekből származó jelalakok a \ref{fig:20}. ábrán láthatóak. A pontos frekvenciát Fourier-transzformációval határoztam meg és a frekvenciateret a \ref{fig:21}. ábrán ismertettem. Az egyes csatornákon kibocsájtott jelek frekvenciája a \ref{table:3}. táblázatban látható.

\begin{center}
\begin{tabular}{|c|c|}
\hline
Csatorna & Frekvencia [Hz] \\ \hline \hline
4        & $1,4 * 10^{5}$  \\ \hline
6        & $1,4 * 10^{5}$  \\ \hline
7        & $9,5 * 10^{4}$  \\ \hline
8        & $1,4 * 10^{5}$  \\ \hline
\end{tabular}
\captionof{table}{A jelgenerátor által kibocsájtott frekvenciák az általunk vizsgált csatornákon.} \label{table:3}
\end{center}
A táblázatból kiolvasható, hogy effektíve csak $2$ csatornát tudtunk használni a $4$ helyett, mely nagyban megnehezítette (sőt egy feladatban lehetetlenné tette) a mérési eredmények kiértékelését. \par
A rezonanciafrekvenciás gerjesztés célja, hogy megtaláljuk azokat a $B$ értékeket, melyek adott $\nu$ frekvencia mellett teljesítik a II. részben tárgyalt rezonanciafeltételt. A mágneses indukció egy szinuszos jel esetén az alábbi módon írható le:

\begin{equation}
B \left( t \right)
=
B_{0} + B_{1} * \sin \left( \omega t \right)
\end{equation}
Ahol $B_{0}$ és $B_{1}$ konstans paraméterek. Ha ekkor $B_{0} \mu_{B} g_{F} = h \nu$, akkor a $B \left( t \right)$ szinuszhullám periódusonként pontosan félperiódusonként metszi a rezonanciafeltételt jelentő $h \nu$ konstans egyenest. A Helmholtz-tekercsre adott áramerősség változtatásával, a feladatunk volt kimérni ezen rezonanciákhoz tartozó $B_{0}$ mágneses indukciókat. Minden frekvenciabeállítás mellett $2$ különböző $B_{0}$ értéknél várható erősítés, melynek oka, hogy a minta egyaránt tartalmaz $^{85}$Rb és $^{87}$Rb izotópokat, melyek eltérő magspinnel és így eltérő $g_{F}$ faktorral rendelkeznek, mely a rezonancia helyét így megváltoztatja. Az optikai pumpálási kísérletek során használt Helmholtz-tekercs középpontjában mérhető mágnes indukció nagysága és a rajta átvezetett áram között az összefüggés az alábbi \citep{MIT_OPU}:

\begin{equation} \label{eq:5}
B
=
\frac{\mu_{0} R^{2} N I}{\left[R^{2} + \left( R/2 \right)^{2} \right]^{3/2}}
=
\frac{8 \mu_{0} N I}{\sqrt{125} R}
\end{equation}
Esetünkben a tekercsmenetek száma $N = 80$, míg sugaruk $R = 0,0895$ m volt.

\subsection{A Föld mágneses terének meghatározása}
Ahhoz, hogy a Föld mágneses teréből származó faktor kiejtsük és kimérjük, a $4$ frekvencián történő méréssort a tekercseken átfolyó áram mindkét polarizációjában megmértük. Mivel a kísérleti összeállítás széndákosan úgy volt kialakítva, hogy a létrejövő $\boldsymbol{B}$ indukcióvektor az É-D-i iránnyal párhuzamos legyen, így egyik esetben azt a Föld mágneses tere erősítette, invertált polarizáció esetén pedig azonos erősséggel gyengítette. Jelöljük az egyik polarizációs esetet egy \q{$+$}, míg az ezzel ellentétes irányút \q{$-$} indexxel. Ekkor minden frekvencián az első esetben $B_{0} = B_{+}$ helyen lesz észlelhető a rezonancia, míg inverz esetben $B_{0} = B_{-}$ indukció mellett. A kettőből megkapható minden egyes esetre a Föld mágneses terének $B_{F}$ értéke, valamint az effektív, \q{valódi} $B_{0}$ érték, melyet a kísérlet során valójában mérnünk kéne, ha a Föld tere nem lenne befolyással rá, és melyet jelöljünk $B^{\ast}$-al:

\begin{equation} \label{eq:6}
B_{F}
=
\frac{B_{+} - B_{-}}{2}
\end{equation}
\begin{equation} \label{eq:7}
B^{\ast}
=
\frac{B_{+} + B_{-}}{2}
\end{equation}
Az egyik polarizáció mellett lemért jelalakok a \ref{fig:22}. ábrán láthatóak. Az ábrán miden sor 1-1 frekvenciacsatornát képvisel, míg a két oszlop a két különböző rubídiumizotóphoz tartozó rezonanciákat ábrázolja. Végeredményben $16$ különböző $I$ értéket mértünk le, mely mért értékek a \ref{table:4}. táblázatban láthatóak.


\begin{center}
\begin{tabular}{|c|c|c|}
\hline
Csatorna & 1. rezonancia [mA] & 2. rezonancia [mA] \\ \hline \hline
$4_{+}$  & $221$              & $365$              \\ \hline
$6_{+}$  & $220$              & $353$              \\ \hline
$7_{+}$  & $133$              & $228$              \\ \hline
$8_{+}$  & $222$              & $357$              \\ \hline
$4_{-}$  & $298$              & $420$              \\ \hline
$6_{-}$  & $293$              & $429$              \\ \hline
$7_{-}$  & $206$              & $390$              \\ \hline
$8_{-}$  & $298$              & $430$              \\ \hline
\end{tabular}
\captionof{table}{A rezonancia pontjában, a Helmholtz-tekercsbe vezetett áramerősségek. A $+$ és $-$ indexek a két különböző polarizációs esetet hivatottak megkülönböztetni.} \label{table:4}
\end{center}
Ezekből az \ref{eq:5}. egyenlet felhasználásával, $8$ darab $B_{+}$ és ugyanennyi $B_{-}$ indukcióértéket tudtunk számolni. A $B_{+}$ és $B_{-}$ mennyiségeket páronként összevonva az \ref{eq:6}. és \ref{eq:7}. egyenletek alapján végül 8 darab $B_{F}$ és ugyanennyi $B^{\ast}$ értéket kaptunk eredményül. \par
A $B_{F}$ értékek átlagolásából megkaptunk egy becslést a Föld mágneses terére, melynek hibáját azok szórása alapján számoltam. A végleges eredmény

\begin{equation}
\left| B_{F} \right|
=
33,4 \pm 12,3\ \mu \text{T}
\end{equation}
mely -- habár nagy hibával rendelkezik -- beleesik a Földön mérhető, standard $25$ - $65\ \mu \text{T}$ tartományba \citep{2010GeoJI.183.1216F}.

\subsection{A rubídiumizotópok Landé-faktorának és magspinjének mérése}
A \ref{eq:7}. egyenlet alapján kapott $B^{\ast}$ értékek segítségével ábrázoltam a $\mu_{B} * B^{\ast}$ - $h \nu$ teret. Eredményül két, különálló pontcsoportot kaptam, melyek a két rubídiumizotóp alapján különültek el egymástól. Ezekre 1-1 egyeneset tudtam illeszteni, melynek meredeksége a rezonanciafeltétel alapján a két izotóphoz tartozó $g_{F}$ faktorokat adja meg. A műszer hibájából fakadóan az illesztés közel ugyanazt a $g_{F}$ faktort adta vissza, így csupán az egyik izotóphoz tartozó $g_{F}$ és $I$ magspin értékeket tudtam meghatározni. Az illesztés eredménye a \ref{fig:23}. ábrán látható. A két adatsorból végül az \ref{table:5}. táblázatban olvasható eredmények jöttek ki.

\begin{center}
\begin{tabular}{|c|c|c|}
\hline
Mennyiség & 1. izotóp         & 2. izotóp           \\ \hline \hline
$g_{F}$   & $0,448 \pm 0,011$ & $0,4796 \pm 0,0084$ \\ \hline
$I$       & $1,73 \pm 0,04$   & $1,585 \pm 0,028$   \\ \hline
\end{tabular}
\captionof{table}{A $\mu_{B} * B^{\ast}$ - $h \nu$ térben való illesztésből kapott értékek.} \label{table:5}
\end{center}
A $^{85}$Rb izotóp esetén valóban $g_{F} \approx 0,541$ értéket vártunk, melyhez az itt kapottak viszonylag közel esnek. A magspin értékét a következő összefüggés alapján adhatjuk meg:

\begin{equation}
g_{F}
=
\frac{g_{J}}{2I + 1}
\end{equation}
Mivel $g_{J}$ értéke -- a mérés során megbeszéltek alapján -- egzaktul 2, így az $I$ értéke könnyen kifejezhető:

\begin{equation}
I = \frac{1}{g_{F}} - \frac{1}{2}
\end{equation}
A kapott $I$ értékek szintén a \ref{table:5}. táblázatban olvashatóak. Az előbb említett $^{85}$Rb magspinje azonban $5/2$, míg a $^{87}$Rb-hez tartozó magspin $3/2$, mely utóbbi jóval közelebb áll az itt kapott eredményekhez.

\section{Diszkusszió}
Összefoglalóan, a mérési feladataink nagy részét sikeresen el tudtunk végezni. Meghatároztuk mind a pumpálás $\tau$, mind pedig $T_{2}$ karakterisztikus idejét. Sikeresen megmértük a Föld mágneses terének nagyságát, végül megkíséreltük meghatározni a mintában található két rubídiumizotóp $g_{F}$ faktorát és $I$ magspinjét, melyre több-kevesebb sikerrel, de részben jó eredményeket kaptunk. \par
Megjegyzés, hogy a rádiófrekvenciás generátor a $2$ évvel ezelőtti mérések során se működött már jól, azonban akkor még lehetett rajta $3$ különböző frekvenciacsatornát találni, így pontosabb eredmények is születtek az akkori mérésekből. Most effektíve két pontra történt az utolsó feladatban az illesztés, mellyel már nem lehetett egészséges hibahatáron belül megállapítani az illesztett egyenesek meredeksége közti különbséget.

\end{multicols}