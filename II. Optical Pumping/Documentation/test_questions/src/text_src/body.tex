\section{Kérdések kidolgozása}
\begin{enumerate}
    \item Q: Mi az a Boltzmann-eloszlás? Adja meg képlettel is!
    \begin{displayquote}
        A: A Boltzmann-eloszlás egy kanonikus rendszer energiájának valószínűségeloszlása, mely megadja, hogy a rendszer mekkora valószínűséggel tartózkodik egy tetszőleges állapotban adott hőmérséklet és $E$ energia mellett. Az eloszlás képlete a következő:
        \begin{equation}
        P_{i}
        =
        \frac{N_{i}}{N}
        =
        \frac{g_{i} e^{-E_{i} / \left( k_{B} T \right)}}{Z \left( T \right)}
        =
        \frac{g_{i} e^{-E_{i} \beta}}{Z \left( T \right)}
        \end{equation}
Ahol $Z \left( T \right)$ a partíciófüggvény, avagy az állapotösszeg:
		\begin{equation}
		Z_{i}
		=
		\sum_{i} g_{i} e^{-E_{i} / \left( k_{B} T \right)}
		=
		\sum_{i} g_{i} e^{-E_{i} \beta}
		\end{equation}
    \end{displayquote}
	
    \item Q: Szobahőmérsékleten, 1 T-nál sokkal kisebb mágneses mező esetén milyen a Zeeman nívók betöltése?
    \begin{displayquote}
    	A: 
    \end{displayquote}
    
    \item Q: Mi az az indukált emisszió? Milyen összefüggés van a fotonok közt?
    \begin{displayquote}
    	A: Ha a rendszert egy olyan EM térrel sugározzuk meg, melyre teljesül az alábbi rezonanciafeltétel:
    		\begin{equation}
    		h\nu
    		=
    		\Delta E
    		=
    		\mu_{B} g_{F} B
    		\end{equation}
akkor vagy az abszorpció jelensége jön létre, mely a pumpálás irányába történő populációátrendeződést eredményez, vagy indukált emisszió, mely az ellentétes irányba taszít vissza részecskéket.
    \end{displayquote}
    
    \item Q: Miért kell minimum 3 nívó az optikai pumpáláshoz?
    \begin{displayquote}
    	A: Minimum $3$ nívó szükséges ahhoz, hogy populációinverziót tudjunk létrehozni ami az erősítés feltétele is egyben.
    \end{displayquote}
    
    \item Q: A pumpáló fény polarizációja hogyan függ ez össze az elektronok impulzusmomentumának megváltozásával?
    \begin{displayquote}
    	A: $E = hf$ energiaátmenet esetén E energiájú foton hatására ámenettel fotonkibocsájtás történik, amely nem különböztethető meg az eredeti fotontól. Az irány, az $E$ és a fázis is megegyezik ebben az esetben.
    \end{displayquote}
    
    \item Q: Lehet-e árnyékolni statikus mágneses teret néhány mm-es fémlemezekkel ill. Faradaykalitkával? Miért?
    \begin{displayquote}
    	A: Nem, a statikus mágneses tér ezeken szinte csillapítás nélkül áthatol. Ezen vezetőből készült burkok működési elve, hogy elektromos tér hatására a benne levő töltések átrendeződnek, melyek megakadályozzák, hogy a vezető által határolt térfogatban elektromos térerősség jöjjön létre. A mágneses tér egészen máshogy működik, mely ellen így ez a koncepció nem képes védekezni.
    \end{displayquote}
    
    \item Q: Mi az a kettőstörés?
    \begin{displayquote}
    	A: Kettőstörésnek hívjuk azt a jelenséget, mikor bizonyos kristályok a rajtuk áthaladó fénysugarat két darab, egymással merőlegesen polarizált fénysugárra bontják.
    \end{displayquote}
    
    \item Q: Hogyan lehet lineárisan polarizált fényből cirkulárist előállítani?
    \begin{displayquote}
    	A: Kettőstörő kristállyal, amely az egyik komponensben egy $\lambda/4$ nagyságú fázistolást ad.
    \end{displayquote}
    
    \item Q: Mi az a $\lambda/4$-es lemez? Milyen anyagból készül? 
    \begin{displayquote}
    	A: Kettőstörő kristályból készült lemez, mely feladata, hogy lineárisan polarizált fényből cirkulárist állítson elő.
    \end{displayquote}
    
    \item Q: Mire szolgál a mintában lévő puffergáz?
    \begin{displayquote}
    	A: A pumpálás hatásossága (az átmenetek teljes száma) függ a rendszer relaxációjának sebességét leírő $T_{1}$ időállandódótól (ld. 12. kérdés). Ezen időállandó növelésének egyik lehetséges módja egy, a műszert kitöltő, diamágneses puffergáz alkalmazása. Ez nagyban csökkenti az edény falával történő, depolarizációt okozó atomi ütközéseket, míg magával a puffergázzal történő ütközések során a megfelelő polarizációt megmarad.
    \end{displayquote}
    
    \item Q: Mekkora a Zeeman-felhasadás energiája?
    \begin{displayquote}
    	A:
    		\begin{equation}
    		E \left( B \right)
    		=
    		E_{0} + \mu_{B} B \left(m_{L} + 2m_{S} \right).
    		\end{equation}
    \end{displayquote}
    
    \item Q: Mi az a $T_{1}$ relaxáció?
    \begin{displayquote}
    	A: A pumpálás során atomi-, vagy magpolarizáció lép fel, mely a Zeeman-nívók egyensúlytól eltérő betöltöttsége miatt jön létre. Ha a pumpálás abbamarad, akkor termikus relaxáció utján a rendszer az egyensúly felé kezd el közeledni, majd valamilyen karakterisztikus $T_{1}$ időállandó által meghatározott idő után újra beáll az egyensúly.
    \end{displayquote}
\end{enumerate}