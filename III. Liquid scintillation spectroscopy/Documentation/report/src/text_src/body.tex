\selectlanguage{english}
\begin{abstract}
    \noindent Az \emph{Alkalmazott fizikai módszerek laboratórium} harmadik alkalmán a radiokarbon ($^{14}$C izotóp) $\beta$-spektrumát mértük ki folyadékszcintillációs spektroszkópia segítségével. A mérőműszer és a mérési feladatok természetéből fakadóan a labor során elsődlegesen nem aktív mérési munkát végeztünk, hanem megismerkednünk a $\beta$-bomlás azon elméleti alapjaival, melyek az eredmények kiértékeléséhez elengedhetetlenül szükségesek. Levezettük a $\beta$-spektrum kialakulásának egy közelítő, nem-relativisztikus modelljét és meghatároztuk annak átlagos energiáját. A mérés kiértékelése során ellenőriztük a modell helyességét az adatokra történő illesztéssel, valamint hasonlóan vizsgáltuk ezen modell módosított változatát is a Fermi-függvény felhasználásával. Megállapítottuk, hogy a $\beta$-spektrum energiájának várható értéke $\left< E \right> = Q/3$, valamint bebizonyítottuk, hogy $\sqrt{N} \approx \sigma$, ahol $N$ a mérési értékek darabszáma, $\sigma$ pedig azok szórása. Végül a méréshez használt minta megadott $DPM$, és a mérési adatainkból számolt $CPM$ értékek segítségével kiszámítottuk a detektor $\eta$ detektálási hatásfokának mértékét is.
\end{abstract}
\selectlanguage{magyar}

\begin{multicols}{2}
\section{Bevezetés}


\end{multicols}