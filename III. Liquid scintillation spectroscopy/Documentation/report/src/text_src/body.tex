\selectlanguage{english}
\begin{abstract}
    \noindent Az \emph{Alkalmazott fizikai módszerek laboratórium} harmadik alkalmán a radiokarbon ($^{14}$C izotóp) $\beta$-spektrumát mértük ki folyadékszcintillációs spektroszkópia segítségével. A mérőműszer és a mérési feladatok természetéből fakadóan a labor során elsődlegesen nem aktív mérési munkát végeztünk, hanem megismerkednünk a $\beta$-bomlás azon elméleti alapjaival, melyek az eredmények kiértékeléséhez elengedhetetlenül szükségesek. Levezettük a $\beta$-spektrum kialakulásának egy közelítő, nem-relativisztikus modelljét és meghatároztuk annak átlagos energiáját. A mérés kiértékelése során ellenőriztük a modell helyességét az adatokra történő illesztéssel, valamint hasonlóan vizsgáltuk ezen modell módosított változatát is a Fermi-függvény felhasználásával. Megállapítottuk, hogy a $\beta$-spektrum energiájának várható értéke $\left< E \right> = Q/3$, valamint bebizonyítottuk, hogy $\sqrt{N} \approx \sigma$, ahol $N$ a mérési értékek darabszáma, $\sigma$ pedig azok szórása. Végül a méréshez használt minta megadott $DPM$, és a mérési adatainkból számolt $CPM$ értékek segítségével kiszámítottuk a detektor $\eta$ detektálási hatásfokának mértékét is.
\end{abstract}
\selectlanguage{magyar}

\begin{multicols}{2}
\section{Bevezetés}
A $\beta$-bomlás ismerete minden fizikával foglakozó számára alapvetően szükséges kell, hogy legyen. Ez már önmagában jelentős belátást nyújt mind a gyenge kölcsönhatás, mind pedig a neutrínók mibenlétére, mely közül az első az alapvető kölcsönhatások, a második pedig az elemi részekék egyike. Mindkettő az ismert fizikai világot alapvetően meghatározó jelenség és objektum, így ismeretük elengedhetetlen. \par
A mérés során a $^{14}$C -- hétköznapi nevén radiokarbon -- $\beta$-spektrumát vizsgáltuk ún. folyadékszcintillációs spektroszkópia segítségével. A mérés során az alábbi kérdésekre kerestük a választ:
\begin{enumerate}
\item Mennyire követik a nem-relativisztikus, valamint a Fermi-függvénnyel kibővített nem-relativisztikus elméleti modellek a mérés során kimért spektrumot? Mik okozhatják az esetleges eltéréseket?
\item Igaz-e, hogy az energia várható értéke az adott mag $Q$-faktorának harmadával egyenlő, tehát
	\begin{equation}
	\left< E \right> = \frac{Q}{3}
	\end{equation}
\item Igaz-e, hogy a mérési adatok számának gyöke $\left( \sqrt{N} \right)$ közelít-e azok szórásának $\left( \sigma \right)$ értékéhez?
\end{enumerate}

\section{Elméleti alapok}


\section{Kiértékelés}


\section{Diszkusszió}


\end{multicols}