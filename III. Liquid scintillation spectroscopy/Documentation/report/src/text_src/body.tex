\selectlanguage{english}
\begin{abstract}
    \noindent Az \emph{Alkalmazott fizikai módszerek laboratórium} harmadik alkalmán a radiokarbon ($^{14}$C izotóp) $\beta$-spektrumát mértük ki folyadékszcintillációs spektroszkópia segítségével. A mérőműszer és a mérési feladatok természetéből fakadóan a labor során elsődlegesen nem aktív mérési munkát végeztünk, hanem megismerkedtünk a $\beta$-bomlás azon elméleti alapjaival, melyek az eredmények kiértékeléséhez elengedhetetlenül szükségesek. Levezettük a $\beta$-spektrum leírásának egy közelítő, nem-relativisztikus modelljét és kiszámítottuk annak átlagos energiáját. A mérés kiértékelése során ellenőriztük a modell helyességét az adatokra történő illesztéssel, valamint hasonlóan vizsgáltuk ezen modell módosított változatát is a Fermi-függvény felhasználásával. Megállapítottuk, hogy a $\beta$-spektrum energiájának várható értéke $\left< E \right> = Q/3$, valamint bebizonyítottuk, hogy $\sqrt{N} \approx \sigma$, ahol $N$ a mérési értékek darabszáma, $\sigma$ pedig azok szórása. Végül a méréshez használt minta megadott $DPM$, és a mérési adatainkból számolt $CPM$ értékek segítségével kiszámítottuk a detektor $\eta$ detektálási hatásfokának mértékét is.
\end{abstract}
\selectlanguage{magyar}

\begin{multicols}{2}
\section{Bevezetés}
A $\beta$-bomlás ismerete minden fizikával foglakozó számára alapvetően szükséges kell, hogy legyen. Ez már önmagában jelentős belátást nyújt mind a gyenge kölcsönhatás, mind pedig a neutrínók mibenlétére, mely közül az első az alapvető kölcsönhatások, a második pedig az elemi részekék egyike. Mindkettő az ismert fizikai világot alapvetően meghatározó jelenség és objektum, így ismeretük elengedhetetlen. \par
A mérés során a $^{14}$C -- hétköznapi nevén radiokarbon -- $\beta$-spektrumát vizsgáltuk ún. folyadékszcintillációs spektroszkópia segítségével. A mérés során az alábbi kérdésekre kerestük a választ:
\begin{itemize}
\item[--] Mennyire követik a nem-relativisztikus, valamint a Fermi-függvénnyel kibővített nem-relativisztikus és relativisztikus elméleti modellek a mérés során kimért spektrumot? Mik okozhatják az esetleges eltéréseket?
\item[--] Igaz-e, hogy az energia várható értéke az adott mag $Q$-faktorának harmadával egyenlő, tehát
	\begin{equation*}
	\left< E \right> = \frac{Q}{3}\ \text{?}
	\end{equation*}
\item[--] Igaz-e, hogy a mérési adatok számának gyöke $\left( \sqrt{N} \right)$ közelít-e azok szórásának $\left( \sigma \right)$ értékéhez?
\end{itemize}

\section{Technikai leírás}
A mérésben használt minta oldott formában, egy lezárt üvegfiolában helyezkedett el, mely mellett az üvegen belül volt megtalálható a három összetevőből álló szcintillációs \q{koktél} is. Ezt a mintát összerázva helyeztük egy már előre kalibrált szcintillációs számláló két, egymással szemben elhelyezkedő fotoelektron-sokszorozója közé. Minden egyéb feladatot innentől a gép végzett el helyettünk. \par
A $^{14}$C mintával végül összesen 50 darab, egyenként 2 perc hosszú mérést végeztünk, így végeredményben 50 teljes spektrumot kaptunk, melyeket így már statisztikai módszerekkel elemezni tudtunk. A sokcsatornás analizátorra kapcsolt számítógép hibájából fakadóan ezen 50 mérés közül csak 49-ből szereztünk felhasználható eredményeket, azonban szerencsére a kötelező egyetemi labormunkák esetén nem várt el akkora pontosság, hogy ez az apró hiány a végeredményeink minőségét negatívan befolyásolná.

\section{Elméleti alapok}
A mérés -- az I. rész felsorolásában is olvasható -- első számú célja a $\beta$-bomlás Fermi-féle modelljének vizsgálata volt. Ehhez meg kellett értenünk a $\beta$-spektrum lehetséges leírását. Kiindulásként a Fermi-féle aranyszabályt vehetjük, mely az átmeneti valószínűséget adja meg két kvantumállapot között:

\begin{equation}
w_{k \to v}
=
\frac{2 \pi}{\hbar} \left| \left< \Psi_{v} \left| H_{\beta} \right| \Psi_{k} \right> \right|^{2} \varrho_{v} \left( E_{v} \right)
\end{equation}
Ahol a $k$ és $v$ indexek sorrendben a kezdő és végállapotra utalnak.

\subsection{Alkalmazott közelítések}
Hogy a számításainkat el tudjuk végezni, a Fermi-szabály kapcsán 5 különböző közelítéssel élünk, melyből következtetünk majd a $\beta$-bomlás végállapoti állapotsűrűségére. Ezek a közelítések az alábbiak:

\begin{enumerate}
\item Csak megengedett átmenetekről beszélünk. \par
Ez esetben a neutrínó és az elektron teljes pályaperdülete $L = 0$. Ekkor a Fermi-féle aranyszabályban szereplő mátrix elem értéke:

\begin{equation}
\left| \left< \Psi_{v} \left| H_{\beta} \right| \Psi_{k} \right> \right|^{2}
=
\left| H_{kv} \right|^{2}
\approx
\text{const.}
\end{equation}
Továbbiakban ebből levezethető a mátrixelem pontos értéke is, mely indikálni fogja a különbséget a Fermi-- (F) és a Gamow--Teller-típusú (GT) átmenetek között:

\begin{equation}
\left| H_{kv} \right|^{2}
=
\left( g_{V}^{2} M_{F}^{2} + g_{A}^{2} M_{GT}^{2} \right) \frac{1}{V^{2}}
\end{equation}
Ahol $g_{V}$ a vektor típusú, míg $g_{A}$ az axiálvektor típusú kölcsönhatás csatolási állandója.

\item Az elektron, valamint a neutrínó hullámfüggvényét nem-relativisztikus síkhullámként kezeljük. \par
$\beta$-bomlás során a mag vonzása a maghoz közel az elektron hullámfüggvényét valós esetben torzítaná, azonban ebben a közelítésben ettől eltekintünk. Ezt az eltérést a Fermi-függvény bevezetésével korrigáljuk. Ekkor a két részecske hullámfüggvénye kifejezhető az alábbi módokon:

\begin{equation}
\phi_{e} \left( \underline{r}_{e} \right)
=
N_{e} e^{-\frac{i}{\hbar} \underline{p}_{e} \underline{r}_{e}}
\end{equation}
\begin{equation}
\phi_{\nu} \left( \underline{r}_{\nu} \right)
=
N_{\nu} e^{-\frac{i}{\hbar} \underline{p}_{\nu} \underline{r}_{\nu}}
\end{equation}
Az állapotok száma egy fázistérfogatban ekkor könnyen felírható mindkét részecske esetére:
\begin{equation}
dn_{e} \left( E \right)
=
\frac{V * 4 \pi p_{e}^{2} dp_{e}}{h^{2}}
\end{equation}
\begin{equation}
dn_{\nu} \left( E_{\nu} \right)
=
\frac{V * 4 \pi p_{\nu}^{2} dp_{\nu}}{h^{2}}
\end{equation}

\item Az e$^-$ és $\nu$ kirepülési irányát függetlennek vesszük. \par
Ekkor az állapotszám adott fázistérfogatban a következő:
\begin{align}
dn \left( E, E_{\nu} \right)
&=
dn_{e} dn_{\nu}
\propto
p_{e}^{2} dp_{e} p_{\nu}^{2} dp_{\nu}
= \nonumber \\
&=
p_{e}^{2} \frac{dp_{e}}{dE} p_{\nu}^{2} \frac{dp_{\nu}}{dE_{\nu}} dE dE_{\nu}
\end{align}
Itt csak arányosságot írtam fel, az egyenlet jobb oldala valójában egy sok tagból álló konstanssal van megszorozva még.

\item A leánymag visszalökődését elhanyagoljuk. A bomlás $Q$ faktorát ekkor az elektron és neutrínó energiájának összegeként kapjuk:

\begin{equation}
Q = E + E_{\nu}
\end{equation}

\item A neutrínó nyugalmi tömegét zérusnak vesszük. \par
Ekkor a neutrínót ún. \q{ultrarelativisztikus} módon kezeljük, impulzusát ilyenkor a $p_{\nu} = E_{\nu}/c$ képlet adja meg.
\end{enumerate}
Ezek alapján mind a nem-relativisztikus, mind pedig a relativisztikus esetre levezethetjük az állapotsűrűséget és így a $\beta$-bomlás energiaspektrumát, attól függően, hogy az elektron impulzusára a nem-relativisztikus, vagy relativisztikus összefüggést használjuk. Első esetben az impulzus a következő:
\begin{equation}
p_{e}
=
\sqrt{2mE}
\end{equation}
\begin{equation}
\frac{dp_{e}}{dE}
=
\frac{m}{\sqrt{2mE}}
\end{equation}
Míg második esetben az alábbi:

\begin{equation}
p_{e}
=
\frac{1}{c} \sqrt{\left( E + m_{e} c^2 \right)^2 - m_{e}^{2} c^{4}}
\end{equation}
\begin{equation}
\frac{dp_{e}}{dE}
=
\frac{1}{c} \frac{E + m_{e} c^2}{\sqrt{\left( E + m_{e} c^2 \right)^2 - m_{e}^{2} c^{4}}}
\end{equation}
A részletesen számolások az \hyperref[appendix:A]{A.} függelékben találhatóak.

\subsection{Várható energiaérték}

\section{Kiértékelés}


\section{Diszkusszió}


\end{multicols}