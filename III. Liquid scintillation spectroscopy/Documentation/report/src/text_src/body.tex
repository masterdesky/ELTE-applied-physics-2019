\selectlanguage{english}
\begin{abstract}
    \noindent Az \emph{Alkalmazott fizikai módszerek laboratórium} harmadik alkalmán a radiokarbon ($^{14}$C izotóp) $\beta$-spektrumát mértük ki folyadékszcintillációs spektroszkópia segítségével. A mérőműszer és a mérési feladatok természetéből fakadóan a labor során elsődlegesen nem aktív mérési munkát végeztünk, hanem megismerkedtünk a $\beta$-bomlás azon elméleti alapjaival, melyek az eredmények kiértékeléséhez elengedhetetlenül szükségesek. Levezettük a $\beta$-spektrum leírásának egy közelítő, nemrelativisztikus modelljét és kiszámítottuk annak átlagos energiáját. A mérés kiértékelése során ellenőriztük a modell helyességét az adatokra történő illesztéssel, valamint hasonlóan vizsgáltuk ezen modell módosított változatát is a Fermi-függvény felhasználásával. Megállapítottuk, hogy a $\beta$-spektrum energiájának várható értéke $\left< E \right> = Q/3$, valamint bebizonyítottuk, hogy $\sqrt{N} \approx \sigma$, ahol $N$ a mérési értékek darabszáma, $\sigma$ pedig azok szórása. Végül a méréshez használt minta megadott $DPM$, és a mérési adatainkból számolt $CPM$ értékek segítségével kiszámítottuk a detektor $\eta$ detektálási hatásfokának mértékét is.
\end{abstract}
\selectlanguage{magyar}

\begin{multicols}{2}
\section{Bevezetés}
A $\beta$-bomlás ismerete minden fizikával foglakozó számára alapvetően szükséges kell, hogy legyen. Ez már önmagában jelentős belátást nyújt mind a gyenge kölcsönhatás, mind pedig a neutrínók mibenlétére, mely közül az első az alapvető kölcsönhatások, a második pedig az elemi részekék egyike. Mindkettő az ismert fizikai világot alapvetően meghatározó jelenség és objektum, így ismeretük elengedhetetlen. \par
A mérés során a $^{14}$C -- hétköznapi nevén radiokarbon -- $\beta$-spektrumát vizsgáltuk ún. folyadékszcintillációs spektroszkópia segítségével. A mérés során az alábbi kérdésekre kerestük a választ:
\begin{itemize}
\item[--] Mennyire követik a nemrelativisztikus, valamint a Fermi-függvénnyel kibővített nemrelativisztikus és relativisztikus elméleti modellek a mérés során kimért spektrumot? Mik okozhatják az esetleges eltéréseket?
\item[--] Igaz-e, hogy az energia várható értéke az adott mag $Q$-faktorának harmadával egyenlő, tehát
	\begin{equation*}
	\left< E \right> = \frac{Q}{3}\ \text{?}
	\end{equation*}
\item[--] Igaz-e, hogy a mérési adatok számának gyöke $\left( \sqrt{N} \right)$ közelít-e azok szórásának $\left( \sigma \right)$ értékéhez?
\end{itemize}

\section{Technikai leírás}
A mérésben használt minta oldott formában, egy lezárt üvegfiolában helyezkedett el, mely mellett az üvegen belül volt megtalálható a három összetevőből álló szcintillációs \q{koktél} is. Ezt a mintát összerázva helyeztük egy már előre kalibrált szcintillációs számláló két, egymással szemben elhelyezkedő fotoelektron-sokszorozója közé. Minden egyéb feladatot innentől a gép végzett el helyettünk. \par
A $^{14}$C mintával végül összesen 50 darab, egyenként 2 perc hosszú mérést végeztünk, így végeredményben 50 teljes spektrumot kaptunk, melyeket így már statisztikai módszerekkel elemezni tudtunk. A sokcsatornás analizátorra kapcsolt számítógép hibájából fakadóan ezen 50 mérés közül csak 49-ből szereztünk felhasználható eredményeket, azonban szerencsére a kötelező egyetemi labormunkák esetén nem várt el akkora pontosság, hogy ez az apró hiány a végeredményeink minőségét negatívan befolyásolná.

\section{Elméleti alapok}
A mérés -- az I. rész felsorolásában is olvasható -- első számú célja a $\beta$-bomlás Fermi-féle modelljének vizsgálata volt. Ehhez meg kellett értenünk a $\beta$-spektrum lehetséges leírását. Kiindulásként a Fermi-féle aranyszabályt vehetjük, mely az átmeneti valószínűséget adja meg két kvantumállapot között:

\begin{equation} \label{eq:1}
w_{k \to v}
=
\frac{2 \pi}{\hbar} \left| \left< \Psi_{v} \left| H_{\beta} \right| \Psi_{k} \right> \right|^{2} \varrho_{v} \left( E_{v} \right)
\end{equation}
Ahol a $k$ és $v$ indexek sorrendben a kezdő és végállapotra utalnak.

\subsection{Alkalmazott közelítések}
Hogy a számításainkat el tudjuk végezni, a Fermi-szabály kapcsán 5 különböző közelítéssel élünk, melyből következtetünk majd a $\beta$-bomlás $dn\left( E \right) / dE$ végállapoti állapotsűrűségére. Ezek a közelítések az alábbiak:

\begin{enumerate}
\item Csak megengedett átmenetekről beszélünk. \par
Ez esetben a neutrínó és az elektron teljes pályaperdülete $L = 0$. Ekkor a Fermi-féle aranyszabályban szereplő mátrixelem értéke:

\begin{equation} \label{eq:2}
\left| \left< \Psi_{v} \left| H_{\beta} \right| \Psi_{k} \right> \right|^{2}
=
\left| H_{kv} \right|^{2}
\approx
\text{const.}
\end{equation}
Továbbiakban ebből levezethető a mátrixelem pontos értéke is, mely indikálni fogja a különbséget a Fermi-- (F) és a Gamow--Teller-típusú (GT) átmenetek között:

\begin{equation} \label{eq:3}
\left| H_{kv} \right|^{2}
=
\left( g_{V}^{2} M_{F}^{2} + g_{A}^{2} M_{GT}^{2} \right) \frac{1}{V^{2}}
\end{equation}
Ahol $g_{V}$ a vektor típusú, míg $g_{A}$ az axiálvektor típusú kölcsönhatás csatolási állandója.

\item Az elektron, valamint a neutrínó hullámfüggvényét nemrelativisztikus síkhullámként kezeljük. \par
$\beta$-bomlás során a mag vonzása a maghoz közel az elektron hullámfüggvényét valós esetben torzítaná, azonban ebben a közelítésben ettől eltekintünk. Ezt az eltérést a Fermi-függvény bevezetésével korrigáljuk. Ekkor a két részecske hullámfüggvénye kifejezhető az alábbi módokon:

\begin{equation} \label{eq:4}
\phi_{e} \left( \underline{r}_{e} \right)
=
N_{e} e^{-\frac{i}{\hbar} \underline{p}_{e} \underline{r}_{e}}
\end{equation}
\begin{equation} \label{eq:5}
\phi_{\nu} \left( \underline{r}_{\nu} \right)
=
N_{\nu} e^{-\frac{i}{\hbar} \underline{p}_{\nu} \underline{r}_{\nu}}
\end{equation}
Az állapotok száma egy fázistérfogatban ekkor könnyen felírható mindkét részecske esetére:
\begin{equation} \label{eq:6}
dn_{e} \left( E \right)
=
\frac{V * 4 \pi p_{e}^{2} dp_{e}}{h^{2}}
\end{equation}
\begin{equation} \label{eq:7}
dn_{\nu} \left( E_{\nu} \right)
=
\frac{V * 4 \pi p_{\nu}^{2} dp_{\nu}}{h^{2}}
\end{equation}

\item Az e$^-$ és $\nu$ kirepülési irányát függetlennek vesszük. \par
Ekkor az állapotszám adott fázistérfogatban a következő:
\begin{align} \label{eq:8}
dn \left( E, E_{\nu} \right)
&=
dn_{e} dn_{\nu}
\propto
p_{e}^{2} dp_{e} p_{\nu}^{2} dp_{\nu}
= \nonumber \\
&=
p_{e}^{2} \frac{dp_{e}}{dE} p_{\nu}^{2} \frac{dp_{\nu}}{dE_{\nu}} dE dE_{\nu}
\end{align}
Ahol felhasználtam a \ref{eq:6}. és \ref{eq:7}. egyenletekben szereplő azonosságot. Itt csak arányosságot írtam fel, az egyenlet jobb oldala valójában egy sok tagból álló konstanssal van megszorozva még.

\item A leánymag visszalökődését elhanyagoljuk. A bomlás $Q$ faktorát ekkor az elektron és neutrínó energiájának összegeként kapjuk:

\begin{equation} \label{eq:9}
Q = E + E_{\nu}
\end{equation}

\item A neutrínó nyugalmi tömegét zérusnak vesszük. \par
Ekkor a neutrínót ún. \q{ultrarelativisztikus} módon kezeljük, impulzusát és annak energia szerinti deriváltját ilyenkor az alábbi összefüggések adják:

\begin{equation} \label{eq:10}
p_{\nu}
=
\frac{E_{\nu}}{c}
\end{equation}

\begin{equation} \label{eq:11}
\frac{dp_{\nu}}{dE}
=
\frac{1}{c}
\end{equation}

\end{enumerate}
Ezek alapján mind a nemrelativisztikus, mind pedig a relativisztikus esetre levezethetjük az állapotsűrűséget és így a $\beta$-bomlás energiaspektrumát, attól függően, hogy az elektron impulzusára a nemrelativisztikus, vagy relativisztikus összefüggést használjuk. Első esetben az impulzus a következő:
\begin{equation} \label{eq:12}
p_{e}
=
\sqrt{2mE}
\end{equation}
\begin{equation} \label{eq:13}
\frac{dp_{e}}{dE}
=
\frac{m}{\sqrt{2mE}}
\end{equation}
Míg második esetben az alábbi:

\begin{equation} \label{eq:14}
p_{e}
=
\frac{1}{c} \sqrt{\left( E + m_{e} c^2 \right)^2 - m_{e}^{2} c^{4}}
\end{equation}
\begin{equation} \label{eq:15}
\frac{dp_{e}}{dE}
=
\frac{1}{c} \frac{E + m_{e} c^2}{\sqrt{\left( E + m_{e} c^2 \right)^2 - m_{e}^{2} c^{4}}}
\end{equation}
A részletesen számolások az \hyperref[appendix:A]{A.} függelékben találhatóak.

\subsection{Várható energiaérték}
A várható érték definíciója alapján megadható a $\beta$-spektrum energiájának várható értéke az alábbi módon:

\begin{equation} \label{eq:16}
\left< E \right>
=
\int_{0}^{Q} E P \left( E \right) dE
\end{equation}
Ahol $P \left( E \right)$ az adott energiájú átmenet valószínűsége. Ez kiszámítható a $dn \left( E \right)/dE$ állapotsűrűség, teljes spektrumra vett összegével történő normálásával:

\begin{equation} \label{eq:17}
P \left( E \right)
=
\frac{dn \left( E \right) / dE}{\int_{0}^{Q} \left[ dn \left( E \right) / dE \right] dE}
\end{equation}
Az energia várható értéke a nemrelativisztikus esetből kapott számítás alapján:

\begin{equation} \label{eq:18}
\left< E \right>
=
\frac{Q}{3}
\end{equation}
A teljes számítás az \hyperref[appendix:A]{A.} függelékben található.

\section{Kiértékelés}
\subsection{Elméleti modell vizsgálata}
Első feladatunk az $\beta$-spektrumokra történő illesztés és annak vizsgálata volt. Ez explicite a $dn \left( E \right)$ értékek integráltjából kapott $n \left( E \right)$ állapotszámok energiafüggő illesztését takarta jelen esetben. A \ref{fig:3}-\ref{fig:6}. ábrákon ezen illesztések láthatóak, melyek közül a \ref{fig:3}. és \ref{fig:4}. ábra a nemrelativisztikus esetet, míg a \ref{fig:5}. és \ref{fig:6}. ábrák a relativisztikus eset illesztését ábrázolják. A nemrelativisztikus eset első ábrája kivételével mindegyik helyen igénybe vettem a Fermi-függvényt, hogy pontosítsam az illesztést. (Magyarán relativisztikus esetben nem is dolgoztam a Fermi-függvény nélküli verzióval.) \par
Az elméletből azt vártuk, hogy a legrosszabb illesztés a Fermi-függvény nélküli nemrelativisztikus esethez tartozik. Ezt követi a Fermi-függvénnyel bővített verzió, majd a szimpla relativisztikus és legjobb illesztésként a Fermi-függvénnyel bővített relativisztikus formula. Míg szemmel is egyértelműen látható módon a valósághoz elméletben is legközelebb álló, relativisztikus, Fermi-függvénnyel kibővített, \ref{fig:5}. ábrán látható görbe illeszkedik rá legjobban a kimért adatsorra, addig annak hibáját sajnos numerikusan nem tudtam meghatározni. Ennek oka az általam használt \texttt{curve\_fit} függvény hiányossága, mely a \texttt{Python 3.7}-es verziójának \texttt{scipy} könyvtárának része. Ez egy, a nem-lineáris négyzetes hibákat iteratív módon minimalizáló algoritmust takar, mely az optimalizált paraméterek konkrét értékei mellett kiszámolja azok kovarianciamátrixát is, mely diagonálisában található elemek gyökét tekinti az egyes paraméterek hibáinak. Míg a paraméterértékek kiszámítása viszonylag jól működik, addig bonyolultabb esetekben a kovarianciamátrixot -- és így a hibák értékét -- a függvény nem képes meghatározni. Kivételes szerencsénk van, hogy jelen esetben az illesztések közül egyértelműen el lehet dönteni, hogy az említett, \ref{fig:5}. ábrán látható függvény a legjobb, ahogy azt az elmélet alapján el is vártuk. Ennek illesztett görbéje a GitHubomon megtalálható \citep{github}. \par
Az egyes spektrumokon egy darab markáns eltérés látható az elméleti görbétől a $35$ -- $50$ keVee sávban. Ez a jól látható kiugrás a két fotoelektron sokszorozóval rendelkező mérőműszerek koincidencia vizsgálatának karakterisztikájából fakad és tipikusan az ábrákon is szereplő kiemelkedést okoz.

\subsection{A várható energiaérték megmérése}
Második feladatunk annak bizonyítása volt, hogy a $\beta$-spektrum energiájának várható értéke valóban körülbelül $Q/3$. Ehhez felhasználtam a \ref{eq:68}. egyenletet, mely alapján bin szélességű téglalapokra felosztottam a $49$ mérés átlagából kapott görbét, majd azok területét összeadva közelítettem a görbe alatti integrál értékét. Az egyenlet alapján ezt az átlagértékek összegével leosztva, megkapjuk a mérésből számított végeredményünket. A $^{14}C$ esetében $Q=156,5$ keVee érték mellett elméletileg a

\begin{equation}
\left< E \right>_{\text{elm}}
=
\frac{Q}{3}
=
52,1\dot{6}\ \text{keVee}
\end{equation}
értéket várjuk eredményül. A mérésből a fenti számítás után ugyanezen értékre
\begin{equation}
\left< E \right>_{\text{mér}}
\approx
50,013\ \text{keVee}
\end{equation}
eredményt kaptam. Ez $4,13\%$-al tér el az elméleti értéktől, így egyértelműen, hibahatáron belüli jó közelítésnek vehető.

\subsection{A szórás közelítő értéke}
Harmadik és egyben utolsó feladatunk a szórás és a mintavételezési elemszám gyöke közti ekvivalencia feltárása volt. Ezt az egyes energia-\emph{bin}ekbe kerülő elemek szórásának és a \emph{bin}be eső mérési értékek gyökének összehasonlításával vizsgáltam meg. Várakozásaink alapján, ha az adott \emph{bin}-hez tartozó $\sigma$ szórás-, valamint $\sqrt{N}$ darabszámértékeket egymás függvényében ábrázoljuk, akkor egy $45^{\circ}$ egyenest fogunk eredményül kapni. \par
A kapott ponthalmazt a \ref{fig:7}. ábrán vizualizáltam, melyen jól látható a lineáris kapcsolat. Az adatpontokat a $\left[ 0, 1 \right]$ intervallumba normáltam, melyben az összefüggés jóságát vizsgálandó a pontok átlagtól vett négyzetes eltérésük gyökét számítottam ki. Ez az érték $RMSE = 0,047$ lett, mely teljes mértékben elfogadhatónak számít.

\section{Diszkusszió}
Habár az elméleti görbére történő illesztés során a manuális levezetésből kapott eredmény rosszabbul szerepelt, mint a géppel történő szimbolikus számítás, végeredményben nagyon jó közelítését sikerült adnom a $\beta$-spektrum alakjára (ld. \ref{fig:5}. ábra). \par
Ezen kívül a kirótt feladataimat sikeresen elvégeztem, az $\left< E \right> = Q/3$ azonosságot és a $\sigma = \sqrt{N}$ ekvivalenciát bebizonyítottam. (Végül de nem utolsó sorban pedig remélhetőleg egy fél magfizika tételt is megtanultam.)

\end{multicols}