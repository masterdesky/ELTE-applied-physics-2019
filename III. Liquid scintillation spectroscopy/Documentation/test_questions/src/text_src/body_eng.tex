\section{English version}
\begin{enumerate}
    \item Q: Why do we use liquid scintillator for detecting the decay of triton\footnote{Another name for tritium}? 
    \begin{displayquote}
        A: The triton is a $\beta$-decaying nuclei, which decay's products'\footnote{$\beta$-particle} mean free path is simply way too short to reach a regular encapsulated detector from a separated source. Using a liquid scintillator allows us to use liquid samples, which are directly mixed with the scintillation-capable fluid. In this case, the decaying particles are right next to the scintillating particles, thus making it possible to detect decays with lower energies and shorter mean free paths.
    \end{displayquote}
    
    \item Q: What process is called scintillation?
    \begin{displayquote}
        A: The scintillation is the process when the scintillator material emits visible or UV photons just after one radioactive decay. The $\beta$-particle after decay, excites the nearby atoms as it travels away from the decayed nuclei. When they return to their ground state, they emits these UV photons.
    \end{displayquote}
    
    \item Q: What does a liquid scintillator contain?
    \begin{displayquote}
        A: It literally just contains such a liquid, which capable of scintillation. Then, this scintillation is amplified by a photomultiplier tube and converted to electrical pulses, which are finally detected by a multi-channel analyzer and it creates an energy-histogram of these detected events.
    \end{displayquote}
    
    \item Q: What is the light output and how does it change in case of quenching?
    \begin{displayquote}
        A: The ionizing electrons are losing kinetic energy by colliding with the scintillator liquid's particles. This energy -- denoted by $E_{d}$ -- is proportional to the number of excited atoms or molecules in the scintillator, which is also proportional to the number of scintillation photons. This number is called the light output. Measuring the distribution of this quantity gives us insights about the $\beta$-decay of the sample inside the detector. To be more precise, light output is defined as the ratio of the number of total measured photons and number of $1$ keV photons. \par
Quenching slightly "squeezes" the light output distribution along the X-axis, while pushing its maximum value a bit higher.
    \end{displayquote}
    
    \item Q: How do the radiocarbon and the triton decay?
    \begin{displayquote}
        A: By $\beta^{-}$-decay: in their nuclei a neutron decays into a proton, emitting an electron $\left( e^{-} \right)$ and an electron antineutrino $\left( \tilde{\nu}_{e} \right)$. In terms of equations we can describe them  as follows for the $^{3}$H:
        \begin{equation}
        ^{3}\text{H} \to ^{3}\text{He} + e^{-} + \tilde{\nu}_{e}
        \end{equation}
And for the $^{14}$C:
		\begin{equation}
		^{14}\text{C} \to ^{14}\text{Ne} + e^{-} + \tilde{\nu}_{e}
		\end{equation}
    \end{displayquote}
    
    \item Q: How do the radiocarbon and the triton is created in our planet?
    \begin{displayquote}
        A: They're both created in the upper atmosphere by cosmic neutrons hitting $^{14}$N isotopes. If a neutron hits an $^{14}$N nuclei, it can either hit out a proton, and build into the place of it, or it can pull a proton and a neutron out with itself. In the first case the following reaction will take place:
        \begin{equation}
        ^{14}\text{N} + \text{n}^{0} \to ^{14}\text{C} + \text{p}^{+}
        \end{equation}
While in the second case, the cosmic neutron will form a tritium nuclei with the swept away proton and neutron from the $^{14}$N nuclei. In the formalism of chemical equations:
		\begin{equation}
		^{14}\text{N} + \text{n}^{0} \to ^{12}\text{C} + ^{3}\text{H}
		\end{equation}
    \end{displayquote}
    
    \item Q: Why is the beta-spectrum continuous?
    \begin{displayquote}
        A: The main difference of $\beta$-decay from other decays, that at the $\beta$-decay, there are two particles emitted instead of one: one $\beta$-particle and a corresponding neutrino or anti-neutrino. However the total released energy is discrete, it is continuously split between the two emitted particles. Due to the very low mass of the neutrino, only the $\beta$-particles energy could be effectively detected, which spectrum thus indeed will be continuous.
    \end{displayquote}
    
    \item Q: What is the spectrum of the amplitude of the electric pulses during a triton
measurement?
    \begin{displayquote}
        A: 
    \end{displayquote}
    
    \item Q: How can the scintillation light be quenched?
    \begin{displayquote}
        A: Quenching can occur as the result of many processes. Eg. the liquid, or the container glass vial can contain contamination. But there are several ways for this.
    \end{displayquote}
    
    \item Q: How can we determine the efficiency of a triton measurement from the measured
value of the quench?
    \begin{displayquote}
        A: 
    \end{displayquote}
    
    \item Q: Why is the amplitude of the electric signal proportional to the energy deposited in the vial?
    \begin{displayquote}
        A: 
    \end{displayquote}
    
    \item Q: What are the operating purposes of the cocktail ingredients?
    \begin{displayquote}
        A: The largest part of this cocktail is the solvent. It is the medium, where the $\beta$-particle releases its kinetic energy. The solvent then emits photons, which wavelength are not optimal for the photomultiplier tube. To solve this issue, there are two more ingredient mixed into the cocktail, which absorbs these useless photons, and emits them on a much more optimal wavelength, which are now can be used for measurements.
    \end{displayquote}
    
    \item Q: How does the photomultiplier work?
    \begin{displayquote}
        A: 
    \end{displayquote}
    
    \item Q: How can we discriminate the triton counts from the radiocarbon counts?
    \begin{displayquote}
        A: 
    \end{displayquote}
\end{enumerate}