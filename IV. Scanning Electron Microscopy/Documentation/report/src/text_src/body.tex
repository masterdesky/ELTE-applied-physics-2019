\begin{abstract}
    \noindent On the fourth occasion of the \emph{Applied Physics Laboratory} we've learned the basic of the usage of a scanning electron microscope (SEM). The nature of this lecture was purely informative, where we didn't have any complex calculation or measurement task, besides a very short one. Our objective was to learn, how to set up the SEM, to take somewhat higher-quality pictures, which could be further analysed if needed.
\end{abstract}

\begin{multicols}{2}
\section{Introduction}
During the lab, we used a SEM, to make observations of three different samples, each for everyone in our lab group. We used an approximately 30 years old equipment, which could be operated and set up only fully manually. This meant, that we have to set all the settings and adjustments for every picture, without the aid of any kind of automatic corrections.


\section{Observations}
The first sample, a microscopic copper lattice was studied by me. I made my observations only with the detector which is capable of the detection of the secondary electrons. Due to the short duration of the lab, I didn't have enough time to study my sample also with the detection of backscattered electrons. \par
The secondary electrons are created, when the electron beam of the SEM knocks out electrons from the higher atomic levels of the sample, into the direction of the detector. These electrons always have low energy, and originate from the uppermost layers of the sample. The SEM uses a so-called Everhard--Thornley-detector, which is technically a photomultiplier tube with an outer electric field, which collects the scattered secondary electrons. \par
During my measurement I had to take numerous pictures of my sample with different magnitudes, and had to calibrate the images' brightness, contrast, focus, position and the corrections for edge distortions. The $95\%$ of my work was consisted of by setting up the first two of them. The image could be displayed by either the SEM's own oscilloscope-like display, or by an external monitor. The contrast and brightness could be efficiently set using the monitor, while the other parameters using the SEM's own monitor. \par
The rendering of the image on the external monitor was painfully slow, the whole picture was gradually displayed on the monitor line by line over the span of a minute. During this rendering the brightness and contrast could be changed in real time, but the changes only took effect on the currently rendered lines, while the already rendered parts were unchanged. This created an image with horizontal segments of gradually changing contrast and brightness. After the rendering was complete, the rendering program returned the histogram of the images. \par
After slowly setting the right brightness and contrast on a picture, we could save it finally. My pictures could be seen on the figures (\ref{fig:1}) - (\ref{fig:8}), along with their corresponding histograms. In these pictures the  On the copper lattice, there are numerous bright spots and artifacts, which are simply contamination, found on the surface of the sample. Thorough the holes of the lattice, the sample holding plate could be seen, which is clearly also contaminated by dust and other pollution. \par
The second sample - which wasn't calibrated by me - was a human hair. The calibration process was the same, but my lab partner had some problems with the correct setting of brightness and contrast, and not even any of us could help her. Here, only two images with secondary electrons, and one with backscattered electrons were taken. One of them which was taken using secondary electrons, and the backscattered one could be seen along with their corresponding histogram on figures (\ref{fig:9}) - (\ref{fig:12}). The first thing which could be seen on these pictures is the damage of the hair, also the smooth surface along which the hair was cut (probably with a razor). In the pictures of the secondary electrons, this surface is very bright, indicating the presence of some other material, than other parts of the hair. \par
The last sample was calibrated by our third lab partner, which sample was some kind of microchip. Setting the correct brightness and contrast was pretty much smoother in this case. Some numbers, which size was approximately $100\ \mu \text{m}$ could be also seen and clearly read from the surface of the chip. These pictures could be seen on figures (\ref{fig:13}) - (\ref{fig:18}).

\section{Discussion}
We've seen the hard part of the calibration and taking images with a scanning electron microscope, but still managed to make our very first pictures with it, which I think is the main objective of this lab measurement. Hence, we can conclude, that our lab work was successful.

\end{multicols}