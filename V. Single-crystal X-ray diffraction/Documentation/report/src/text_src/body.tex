\selectlanguage{english}
\begin{abstract}
    \noindent Az \emph{Alkalmazott fizikai módszerek laboratórium} ötödik alkalmán különböző, kristályos szerkezetű anyagok rácsának orientációját vizsgáltuk meg Laue-féle röntgendiffrakció alkalmazásával. A méréseink során három különböző egykristály mintát vizsgáltunk, melyek között egy kisméretű bányászott sótömb, egy nagy tisztaságú szilíciumlapka, valamint egy keménységméréshez használt csiszolt gyémánt volt megtalálható.
\end{abstract}
\selectlanguage{magyar}

\begin{multicols}{2}
\section{Bevezetés}
A röntgendiffrakció különböző módszerei az anyag, annak molekuláris szerkezetének nagyságrendjében történő feltérképezését teszi lehetővé, melyet számos tudományterületben széles körűen alkalmaznak. \par
A mérés során az ún. Laue-féle diffrakció módszerét alkalmaztuk, mely tipikusan a feladatunkhoz hasonlóan, egykristályok rácsszerkezetének orientációját feltérképezendő használatos.

\section{Technikai részletek}
A mérés során egy többfunkciós berendezést használtunk, mely mind pordiffrakciós, mind pedig Laue-diffrakciós mérésre alkalmas volt. Az általunk felhasznált lágy röntgensugárzást egy röntgen kisülési cső hozta létre, melyet $40$ keV feszültség és $20$ mA áramerősség alatt működtettünk. A kisülési cső oldalán a sugárzás egy vékony berillium \q{ablakon} keresztül távozott, mely túlhevülését elkerülendő, a rendszert folyamatosan hűtöttük. Túlmelegedés esetén a berilliumlapka könnyen széttörik, mely a kisülési cső részleges tönkremenetelét is jelentené egyben. \par
A mérőműszerből távozó sugárzás irányát egy fluoreszenciás lapka segítségéve kalibráltuk, mely alapján megállapítható volt, hogy a kijövő sugár viszonylag vékony, nagyságrendileg hozzávetőlegesen 2 mm átmérőjű. \par
A mérésben úgynevezett hátsó állású képeket készítettünk, ami azt takarja, hogy a mintáról visszafelé szóródó sugárzást észleltük, melyet egy \q{image plate} lapka használatával fogtunk fel. Végül ezen lap kiolvasásával jutottunk hozzá a kiértékeléshez szükséges adatokhoz.

\section{A kiértékelés menete}
\subsection{Image plate}
Az image plate, kialakításából fakadóan egy digitális tároló eszköz, melyet egy speciális szkennelő eszközzel tudunk kiolvasni. A röntgen besugárzás hatására a lapka érintett területei gerjesztett állapotba kerülnek, melyek a gerjesztéshez használtnál alacsonyabb frekvenciájú sugárzás hatására visszaállnak eredeti állapotukba. Ennek során karakterisztikus sugárzást bocsájtanak ki, melyet pontosan mérhetünk, így kapva egy végleges, digitális képet. Erős fénysugárzás hatására a lapkán található információ teljesen \q{kitörlődik} és felhasználható további mérésekre.

\subsection{OrientExpress}
A rendelkezésre álló képeket egy \q{OrientExpress} nevű szoftverrel dolgoztuk ezután fel, mely képes beazonosítani a mért anyag rácsszerkezetét, adott erősítési pontok megadásának segítségével. A program használata azonban sajnos nem triviális. A képeken található erősítési pontok adott hányada származhat akár a mintatartóról, akár valamilyen szennyeződésből is. Ilyeneket is belevéve a program számára megadott erősítési pontok közé, már maga a rács beazonosításának folyamata is hibás adatokból fog kiindulni, így a kapott eredmények szinte teljesen biztosan nem lesznek megfelelőek. Sokszor megesik azonban, hogy még megfelelő pontok megadásával se kapjuk meg a számunkra szükséges végeredményt, így időt kell fordítani a kiértékelés ezen szakaszára.

\section{a}


\section{Diszkusszió}


\end{multicols}