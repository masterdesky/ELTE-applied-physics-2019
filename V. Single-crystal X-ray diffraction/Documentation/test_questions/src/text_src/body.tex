\section{Magyar verzió}
\begin{enumerate}
	\item Q: Mi a Bragg-törvény?
    \begin{displayquote}
        A: A konstruktív interferencia feltételét megadó összefüggés egy kristálysíkra $\theta$ szögben érkező, $\lambda$ hullámhosszú fény esetében, ahol a párhuzamos kristálysíkok távolsága $d$. A Bragg-feltétel az alábbi:
        \begin{equation}
		2 d \sin \left( \theta \right) = \lambda
        \end{equation}
    \end{displayquote}
    \item Q: Mi a zónatengely?
    \begin{displayquote}
        A: Azok a hálózati síkok, amelyek párhuzamosak egy közös iránnyal egy zónát alkotnak, a közös irány pedig a zónatengely. Ez a tengely a zónában levő síkok normálvektorára merőleges.
    \end{displayquote}
    \item Q: Milyen sugárzást használunk?
    \begin{displayquote}
        A: Röntgensugárzást
    \end{displayquote}
\end{enumerate}