\selectlanguage{english}
\begin{abstract}
    \noindent Az \emph{Alkalmazott fizikai módszerek laboratórium} hatodik és egyben utolsó alkalmán egy Cu anyagdarabon végzett röntgen vonalprofil analízis mérés eredményeinek kiértékelését végeztük el. A labor során 
\end{abstract}
\selectlanguage{magyar}

\begin{multicols}{2}
\section{Bevezetés}
A röntgendiffrakció különböző módszerei az anyag, annak molekuláris szerkezetének nagyságrendjében történő feltérképezését teszi lehetővé, melyet számos tudományterületben széles körűen alkalmaznak. \par
A mérés során az ún. Laue-féle diffrakció módszerét alkalmaztuk, mely tipikusan a feladatunkhoz hasonlóan, egykristályok rácsszerkezetének orientációját feltérképezendő használatos.

\section{Technikai részletek}
A mérés során egy többfunkciós berendezést használtunk, mely mind pordiffrakciós, mind pedig Laue-diffrakciós mérésre alkalmas volt. Az általunk felhasznált lágy röntgensugárzást egy röntgen kisülési cső hozta létre, melyet $40$ keV feszültség és $20$ mA áramerősség alatt működtettünk. A kisülési cső oldalán a sugárzás egy vékony berillium \q{ablakon} keresztül távozott, mely túlhevülését elkerülendő, a rendszert folyamatosan hűtöttük. Túlmelegedés esetén a berilliumlapka könnyen széttörik, mely a kisülési cső részleges tönkremenetelét is jelentené egyben. \par
A mérőműszerből távozó sugárzás irányát egy fluoreszenciás lapka segítségéve kalibráltuk, mely alapján megállapítható volt, hogy a kijövő sugár viszonylag vékony, nagyságrendileg hozzávetőlegesen 2 mm átmérőjű. \par
A mérésben úgynevezett hátsó állású képeket készítettünk, ami azt takarja, hogy a mintáról visszafelé szóródó sugárzást észleltük, melyet egy $9,4$ cm $\times$ $9,4$ cm méretű \q{image plate} lapka használatával fogtunk fel. Végül ezen lap kiolvasásával jutottunk hozzá a kiértékeléshez szükséges adatokhoz.

\section{A kiértékelés menete}
\subsection{Image plate}
Az image plate (ún. PSP plate -- photostimulable phosphor plate), kialakításából fakadóan egy digitális tároló eszköz. Röntgenbesugárzás hatására a lapka érintett területein található fotoérzékeny anyag (sok esetben európium) atomjai gerjesztett állapotba kerülnek és ott ragadnak. Ezek a gerjesztéshez használtnál alacsonyabb frekvenciájú, gerjesztésre már nem képes sugárzás hatására visszaállnak eredeti állapotukba. Ekkor karakterisztikus (európium esetén $400$ nm hullámhosszú) sugárzást bocsájtanak ki, mely fotonok intenzitását pontosan megmérhetjük, ezekből kapva egy végleges, digitális képet. Végezetül erős, megvilágított szobában található fénnyel azonos intenzitású megvilágítás hatására a lapkán található információ teljesen \q{kitörlődik} és felhasználható további mérésekre.

\subsection{OrientExpress}
A rendelkezésre álló képeket egy \q{OrientExpress} nevű szoftverrel dolgoztuk ezután fel, mely képes beazonosítani a mért anyag rácsszerkezetét, adott erősítési pontok, képeken történő megadásának segítségével. A program használata azonban sajnos nem triviális. A képeken található erősítési pontok adott hányada származhat akár a mintatartóról, akár valamilyen szennyeződésből is. Ilyen \q{hibás} értékeket is belevéve a program számára megadott pontok közé, megzavarhatja a rácsszerkezet beazonosításának folyamatát, így a kapott eredmények szinte teljesen biztosan nem lesznek megfelelőek. Sokszor megesik azonban, hogy még megfelelő pontok megadásával se kapjuk meg a számunkra szükséges végeredményt, így időt kell fordítani a kiértékelés ezen szakaszára. \par
A szoftver segítségével a PSP lapkáról beolvasott képek kiértékelésének menete az adott anyagra jellemző szerkezeti adatok megadásával kezdődött. Ezeket a laborvezetőtől készen megkaptuk, melyet az OrientExpress számára meg is adhattunk bemeneti adatként. Következő lépésben az adott mintához tartozó képeket töltöttük be a programba, amiket több dolgot kellett megadjunk:
\begin{enumerate}
	\item A röntgen nyaláb, képen látható pozícióját
	\item A kép nagyságrendjét cm-ben
	\item Néhány tetszőlegesen választott erősítési pont pozícióját
\end{enumerate}
Ezeket mind egérmutatóval történő kattintás útján vehettük fel bemeneti adatként. Ezt követően ezen adatok alapján a program megpróbálta meghatározni a rácsszerkezetet leíró Miller-indexeket. A legrizikósabb lépés ez volt, ugyanis az előzőleg kiválasztott pontok függvényében minden esetben más és más Miller-indexeket ajánlott fel a szoftver megoldásként. Végezetül egy sztereografikus megjelenítési módba váltva be tudtuk állítani a diffrakciós kép pozícióját olyan formában, hogy az $x$-$y$-$z$ tengelyek közül tetszőleges kettő a kép síkjára párhuzamos, míg a harmadik arra merőleges legyen. Ismerve az ezen forgatáshoz szükséges szögek értékeit, ez pontosan megadta a rácsszerkezet orientációját a röntgennyalábhoz viszonyítva. \par
Számos próbálgatás után végül mind a három mintára megkaptuk az azok rácsszerkezetében található atomok helyzetét. A Laue-felvételek sorrendben a szilícium-, a gyémánt-, valamint a sómintához megtalálhatóak az (\ref{fig:1})-es, (\ref{fig:3})-as és (\ref{fig:5})-ös ábrákon, míg azonos sorrendben a sztereografikus felvételek a (\ref{fig:2})-es, (\ref{fig:4})-es és (\ref{fig:6})-os ábrákon láthatóak.

\section{Eredmények}
A szilícium lapka orientációjának megállapításához az alábbi Miller-indexszel rendelkező erősítési pontokat választottam ki:

\begin{center}
\begin{tabular}{|c|c|c|c|c|}
\hline
\multicolumn{5}{|c|}{Miller-indexek} \\ \hline \hline
h & -5 & -5 & -2 & -3 \\
k & -5 & -3 & -1 & -3 \\
l & -3 & -5 & -1 & -5 \\
\hline
\end{tabular}
\captionof{table}{A szilíciumlap kristályrács orientációjának kiszámításához felhasznált Miller-indexű pontok.} \label{table:1}
\end{center}
A kapott szerkezetet sztereografikus projekcióban ábrázolva beforgattam a megfelelő pozícióba. A forgatási szögek az alábbiak voltak:

\begin{center}
\begin{tabular}{|c|c|c|}
\hline
\multicolumn{3}{|c|}{Forgatási szögek} \\ \hline \hline
x            & y             & z            \\ \hline
$35^{\circ}$ & $-35^{\circ}$ & $45^{\circ}$ \\
\hline
\end{tabular}
\captionof{table}{A szilíciumlapka rácsának sztereografikus módban történő forgatási szögeinek értéke. Az egyes oszlopok az adott tengely körül történő forgatást jelentik.} \label{table:2}
\end{center}
Hasonlóan az előzőhöz a gyémántdarab orientációjának vizsgálatához az alábbi pontokat választottam:

\begin{center}
\begin{tabular}{|c|c|c|c|c|}
\hline
\multicolumn{5}{|c|}{Miller-indexek} \\ \hline \hline
h & -5 & -3 & -7 & -3 \\
k & -3 & -3 & -1 & -6 \\
l & -2 & -1 & -6 & -5 \\
\hline
\end{tabular}
\captionof{table}{A gyémántrács orientációjának kiszámításához felhasznált Miller-indexű pontok.} \label{table:3}
\end{center}
Sztereografikus pozícióban alábbi szögű forgatásokat végeztem el, hogy hasonló

\begin{center}
\begin{tabular}{|c|c|c|}
\hline
\multicolumn{3}{|c|}{Forgatási szögek} \\ \hline \hline
x            & y             & z            \\ \hline
$54^{\circ}$ & $-35^{\circ}$ & $45^{\circ}$ \\
\hline
\end{tabular}
\captionof{table}{A gyémántrács sztereografikus módban történő forgatási szögeinek értéke. Az egyes oszlopok az adott tengely körül történő forgatást jelentik.} \label{table:4}
\end{center}
Végül a sókristályra kapott értékek a fentiekhez hasonlóan a következők:

\begin{center}
\begin{tabular}{|c|c|c|c|c|}
\hline
\multicolumn{5}{|c|}{Miller-indexek} \\ \hline \hline
h & -4 & -4 & -2 & -1 \\
k & -1 & -2 & -1 &  0 \\
l &  0 & -1 &  0 &  0 \\
\hline
\end{tabular}
\captionof{table}{A sókristály orientációjának kiszámításához felhasznált Miller-indexű pontok.} \label{table:5}
\end{center}
\begin{center}
\begin{tabular}{|c|c|c|}
\hline
\multicolumn{3}{|c|}{Forgatási szögek} \\ \hline \hline
x            & y             & z            \\ \hline
$133^{\circ}$ & $0^{\circ}$ & $1^{\circ}$   \\
\hline
\end{tabular}
\captionof{table}{A sókristály sztereografikus módban történő forgatási szögeinek értéke. Az egyes oszlopok az adott tengely körül történő forgatást jelentik.} \label{table:6}
\end{center}

\section{Diszkusszió}
Míg szemmel vizsgálva a gyémánt és a szilícium esetén elég pontosan sikerült megállapítani a rácsszerkezetben található atomok pozícióját, addig a a sóminta esetén csak egy közelítő becslést sikerült az OrientExpress szoftverrel megadni. A mérési eredmények pontossága tovább javítható lenne újabb, jobb minőségű felvételek készítésével, vagy még inkább korszerűbb elemző szoftverek alkalmazásával. Sajnos a labormérés keretében ezek nem álltak rendelkezésünkre, így itt az általam elért legjobb, azonban objektíve nem teljesen pontos eredményeket kellett közölnöm.

\end{multicols}