\selectlanguage{english}
\begin{abstract}
    \noindent Az \emph{Alkalmazott fizikai módszerek laboratórium} hatodik és egyben utolsó alkalmán egy Cu anyagdarabon végzett röntgen vonalprofil analízis mérés eredményeinek kiértékelését végeztük el. A labor során elkészítettük a minta esetében a Williamson-Hall, valamint módosított Williamson-Hall ábráját, valamint meghatároztuk mikroszerkezeti paramétereit a CMWP módszer segítségével.
\end{abstract}
\selectlanguage{magyar}

\begin{multicols}{2}
\section{Bevezetés}
A röntgensugárzással és annak diffrakciójának vizsgálatával történő analízis különböző módszerei az anyag, annak molekuláris szerkezetének nagyságrendjében történő feltérképezését teszi lehetővé, melyet számos tudományterületben széles körűen alkalmaznak. \par
A mérés során az ún. röntgen vonalprofil analízis módszerével ismerkedtünk meg, mely az anyag kristálysíkjai által létrehozott diffrakciós csúcsok vizsgálatával próbál az anyag mikroszerkezetére vonatkozó következtetéseket levonni.

\section{Technikai részletek}
\subsection{Mérési összeállítás}
A labor során a mérőműszerrel nem kellett dolgoznunk, azt csak megtekintettük a laborvezető kíséretében, aki röviden ismertette annak működését. \par
Egy valódi mérés során a mintát egy precíziós goniométerre szerelt állványra van rögzítve. Ezzel nagy pontossággal beállítható a minta és a röntgensugár relatív pozíciója, mellyel különböző rácssík-seregek vizsgálata válik a minta precíz forgatásával lehetségessé. A mintából diffraktálódó röntgensugárzást egy ún \q{image plate} segítségével fogjuk fel, mely az adott mérés természetéből fakadóan lehet lapos, vagy körívszerűen hajlított is. Jelen esetben egy utóbbi image plate segítségével végzett mérés kiértékelésén dolgoztunk. \par
A mérés során a mintát lassan körbeforgatják és így a beesési szög függvényében kimérhetővé válnak az diffrakciós csúcsok maximumai és azok szélessége, mely információkból következtethetünk a rácssíkok elhelyezkedésére és a vizsgált anyag mikroszerkezetére.

\subsection{Image plate}
Az image plate (ún. PSP plate -- photostimulable phosphor plate), kialakításából fakadóan egy digitális tároló eszköz. Röntgenbesugárzás hatására a lapka érintett területein található fotoérzékeny anyag (sok esetben európium) atomjai gerjesztett állapotba kerülnek és ott ragadnak. Ezek a gerjesztéshez használtnál alacsonyabb frekvenciájú, gerjesztésre már nem képes sugárzás hatására visszaállnak eredeti állapotukba. Ekkor karakterisztikus (európium esetén $400$ nm hullámhosszú) sugárzást bocsájtanak ki, mely fotonok intenzitását pontosan megmérhetjük, ezekből kapva egy végleges, digitális képet. Végezetül erős, megvilágított szobában található fénnyel azonos intenzitású megvilágítás hatására a lapkán található információ teljesen \q{kitörlődik} és felhasználható további mérésekre.

\section{A kiértékelés menete}
A kiértékelést tulajdonképpen több, célirányosan megírt, az ELTÉ-n elérhető szoftver használatával végeztük el. Első lépésben az image plate-ről beolvasott képeket alakítottuk át számunkra használható formába. Az előhívott képeken $7$ különböző rácssíkból származó diffrakciós csúcs volt észlelhető. Ezeket a program és a Cu-re vonatkozó diffrakciós táblázat segítségével azonosítottuk és egy szög-intenzitás függvénnyé átalakítva azokat összefűztük. Továbbiakban az Origin szoftver segítségével kiszámítottuk az egyes csúcsok integrálját és ezzel meghatároztuk a $\Delta \left( 2 \Theta \right)$ értéket, melyet a Williamson-Hall ábra elkészítéséhez használtunk fel. Az image plate szakaszosan történő kiolvasásából kapott nyers eredmények a (\ref{fig:1})-es ábrán találhatóak. Az ábrából látszik többek között az is, hogy az itt használt szoftver a különböző beolvasásokat egy közös háttérre helyezi, majd azok között a hiányzó részeket megfelelő módon kiegészíti.

\subsection{Williamson-Hall ábrázolás}
A szimpla Williamson-Hall ábrázolás során az egyes csúcsok FWHM értékét (félértékszélesség) ábrázoljuk a diffrakciós vektor hosszának függvényében. Ennek az ábrázolásnak a sajátossága, hogy a diffrakciós csúcsok kiszélesedését kizárólag a szemcseméretből származó hatást vesszük figyelembe. \par
Ehhez előbb ki kell számítanunk a félértékszélességet:

\begin{equation}
\text{FWHM}
=
\frac{\cos \left( \Theta \right) * \Delta \left( 2 \Theta \right)}{\lambda} 
\end{equation}
Ahol $\lambda$ a diffrakcióhoz használt röntgensugárzás hullámhossza, jelen esetben

\begin{equation*}
\lambda
=
0.15406\ \text{m}^{-9}
\end{equation*}
Az egyes $\Theta$ szögek értékét nem mi határoztuk meg, hanem egy kész mérésből származó táblázat alapján azonosítottuk be őket. Ez alapján az egyes $hkl$ rácssíkokhoz tartozó diffrakciós csúcsok pozíciói az alábbiak voltak:

\begin{center}
\begin{tabular}{|c|c|c|c|}
\hline
$2 \Theta$ [$^{\circ}$] & $h$ & $k$ & $l$ \\ \hline\hline
$43.297$                & 1   & 1   & 1 \\
$50.433$                & 2   & 0   & 0 \\
$74.13$                 & 2   & 2   & 0 \\
$89.931$                & 3   & 1   & 1 \\
$95.139$                & 2   & 2   & 2 \\
$116.919$               & 4   & 0   & 0 \\ \hline
\end{tabular}
\end{center}
A kapott FWHM adatokból végül megkapható a diffrakciós vektor hossza az egyes rácssíkokhoz tartozó diffrakciók esetében:

\begin{equation}
\left| \boldsymbol{g} \right|
=
\frac{2 \sin \left( \Theta \right)}{\lambda}
\end{equation}
Ezek függvényében ábrázolhatjuk az egyes FWHM értékeket, melyre megkapjuk az RVA mérés Williamson-Hall ábráját, mely a (\ref{fig:3})-as képen látható. Ezt tovább pontosítandó, figyelembe kell vennünk a diffrakciós csúcsok, diszlokációkból eredő szélesedését is, ezzel megalkotva a módosított Williamson-Hall ábrázolást. Ehhez a $g^{2} \overline{C}$ érték függvényében ábrázoljuk az FWHM értéket, ahol $\overline{C}$ az átlagos kontrasztfaktor. Ez a faktor köbös rácsok esetére kiszámítható az alábbi módon:

\begin{equation}
\overline{C}
=
\overline{C}_{h00} * \left( 1 - qH^{2} \right)
\end{equation}
ahol $\overline{C}_{h00}$ a $hkl = h00$ rácssíkokhoz tartozó átlagos kontrasztfaktor, $H^{2}$ pedig a $hkl$ rácssíkok Miller-indexeiből kapott érték:

\begin{equation}
H^{2}
=
\frac{h^{2} k^{2} + h^{2} l^{2} + k^{2} l^{2}}{\left( h^{2} + k^{2} + l^{2} \right)^{2}}
\end{equation}
Végül $q$ az anyag rugalmas állandóitól, valamint a benne található diszlokációk típusától függő kontraszt. Mind a $\overline{C}_{h00}$, mind pedig a $q$ mennyiségek értékét a laborleírásban található táblázatból néztük ki, melyek az alábbiak voltak a Cu esetére:
\begin{center}
\begin{tabular}{|c|c|c|}
\hline
Diszl. típusa & $\overline{C}_{h00}$ & $q$    \\ \hline\hline
Él            & $0.3076$             & $1.67$ \\
Csavar        & $0.3029$             & $2.33$ \\ \hline
Átlag         & $0.30525$            & $2$    \\ \hline
\end{tabular}
\end{center}
A $g^{2} \overline{C}$ -- FWHM függvényt háromféle módon is ábrázoltuk. Első esetben kizárólagosan az éldiszlokációkra vonatkozó $\overline{C}_{h00}$ és $q$, majd a csavardiszlokációkra vonatkozó értékek függvényében ábrázoltuk a félértékszélességet. Végül ezen értékek átlagára számítottuk ki a $\overline{C}$ értéket és azt használtuk fel az ábrázoláshoz. Az így kapott függvény pontokra egy másodrendű polinom illeszthető, mely illesztésből megbecsülhető az adott mérés jósága. A módosított Williamson-Hall ábrák az említett sorrendben a (\ref{fig:4})-(\ref{fig:6}) képeken láthatóak.

\subsection{CMWP módszer}
A CMWP illesztést egy újabb szoftver segítségével végeztük el, mely szinte interakció nélkül, egyszerűen néhány kezdeti paraméter megadása után elvégezte a teljes kiértékelést és az illesztést. Eredményként megkaptunk számtalan mikroszerkezeti- és diffrakciós profil illesztési paramétert. A mért adatsorra illesztett görbe, valamint ezen kettő különbségének függvénye a (\ref{fig:7})-es ábrán látható. Végeredményben az alábbi mikroszerkezeti paramétereket adta vissza az illesztés:

\begin{center}
\begin{tabular}{|c|c|}
\hline
Paraméter & Érték       \\ \hline \hline
\multicolumn{2}{|c|}{\textbf{Méret-paraméterek}} \\ \hline
m              & 6.539955 nm               \\
$\sigma$       & 0.915908                  \\
d              & 137619 nm                 \\
L$_{0}$        & 4268.16 nm                \\ \hline
\multicolumn{2}{|c|}{\textbf{Deformáció-paraméterek}} \\ \hline
q              & 1.96102                   \\
$\varrho$      & 0.00061786 nm$^{-2}$      \\
R$_{e}^{\ast}$ & 48.9026 nm                \\
M$^{\ast}$     & 1.21556                   \\
\hline
\end{tabular}
\end{center}

\section{Diszkusszió}
A labor során megismerkedtünk a röntgen vonalprofil analízis mérési módszerével és sikeresen megvalósítottuk a Williamson-Hall ábrázolást, valamint a CMWP-illesztést.

\end{multicols}