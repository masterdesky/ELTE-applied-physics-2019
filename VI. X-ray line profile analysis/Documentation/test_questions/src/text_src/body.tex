\section{Magyar verzió}
\begin{enumerate}
	\item Q: Mi a különbség a szemcseméretből és a deformációból származó intenzitásprofil rendfüggése között?
    \begin{displayquote}
        A: Míg gömb alakú szemcsék esetén a \q{szemcseméret} vonalprofil alakja nem függ a $\vec{g}$ vektor hosszától, tehát ún. rendfüggetlen, addig a \q{deformációs} vonalprofil szélessége növekszik a $\vec{g}$ növekedésével, tehát ún. rendfüggő.
    \end{displayquote}
    \item Q: A $hkl$ indexű, szemcseméretből és deformációból származó intenzitásprofil kiszélesedését a szemcsék és a deformáció milyen irányú kiterjedése okozza?
    \begin{displayquote}
        A: A szemcséket a diffrakciós vektorral párhuzamos oszlopokba felosztva, ezen oszlopok a $hkl$ rácssíkokra merőlegesek lesznek. Az intenzitásprofil szélességét az ezen merőleges irányban található reflektáló szemcsék mérete fogja meghatározni. Minél kisebb a szemcsék mérete, annál szélesebb a diffrakciós csúcs. A rácsdeformációk esetén hasonlóan, a $hkl$ síkokra merőleges deformációk határozzák meg a vonalprofil kiszélesedését. Minél nagyobb a deformáció mértéke, annál szélesebbé válik a diffrakciós csúcs.
    \end{displayquote}
    \item Q: Milyen paramétereket lehet meghatározni a CMWP-módszerrel?
    \begin{displayquote}
        A: Mikroszerkezeti paramétereket. A laborleírás alapján ezek a következőek:
        \begin{enumerate}
        \item $m$, a szemcseméret-eloszlás középértéke (mediánja), ami az a méret, aminél kisebb és nagyobb szemcsék egyenlő valószínűséggel találhatók az eloszlásban. Dimenziója általában: nm.
		\item $\sigma$, a szemcseméret-eloszlás szórása (varianciája), ami az eloszlás szélességére jellemző. Dimenziótlan mennyiség.
		\item $\varrho$, a diszlokációsűrűség. Dimenzója általában: $1/m^{2}$.
		\item $R_{e}^{\ast}$, a diszlokációk effektív külső levágási sugara. Dimenzója általában: nm. Az $R_{e}^{\ast}$ helyett inkább a dimenziótlan $M^{\ast} = R_{e}^{\ast} \varrho^{1/2}$  mennyiséget szokták használni, amit diszlokáció elrendeződési paraméternek neveznek.
		\item $q$, a diszlokációk típusára jellemző paraméter. Köbös anyag esetén els+sorban az él/csavar jelleget adja meg. Dimenziótlan mennyiség. 
        \end{enumerate}
    \end{displayquote}
\end{enumerate}